\chapter{Benchmarks}
\label{chap:walktrough}

In the previous chapter, we explored the implementation of the \emph{InterfaceSmith} programming system, described the UI and features, and described the code generation capabilities.
This chapter evaluates our \emph{InterfaceSmith} prototype system on three tasks: a simple \emph{TO-DO list} application and the \emph{Counter} and \emph{Temperature Converter} from the \citet{7GUIs-web}.
Our primary goal is to assess whether we can implement these tasks according to their
specifications by using only our prototype system and to determine if our approach successfully
reduces the amount of code that needs to be written,
as per the definition of low-code programming systems by \citet{Pinho_Aguiar_Amaral_2023}.

\section{Methodology}
We will evaluate our prototype system based on the following criteria:
\begin{itemize}
	\item \textbf{Successful implementation of all specified functionality:}
	      We must implement the exact functionality described by each task.
	      If we cannot implement a specific task as specified, we can reason about our system's limitations and potentially
	      identify problems with our implementation or design.

	\item \textbf{Number of lines of code written:}
	      If a referential solution exists, we will compare the number of lines of code written using our system versus the provided referential solution.
	      We will only consider \emph{physical lines of code} as defined by~\citet{Park_1992}.
	      Because our system requires concrete data before we can begin building the desired application, we will also include the number of lines of code needed to be written for the data preparation.
	      We will also ignore all lines of code related to styling the elements and focus only on implementing the specified functionality and UI structure.
\end{itemize}

\noindent For each task, we will first describe the requirements, such as the intended functionality.
After that, we will outline the implementation process using the \emph{InterfaceSmith} system.
We will then present the resulting application or describe what functionality cannot be implemented.
Lastly, we will analyze the number of lines of code written.


\section{TO-DO list application}
The first task is creating a simple TO-DO list application inspired by the \citet{TodoMVC} benchmark.
As our prototype system focuses mainly on exploring the data-driven UI creation and provides limited options for modifying the application's functionality,
we will implement a subset of the functionality specified by the \citet{todo-spec}.
As we will implement a subset of the specified functionality, we will not compare the resulting number of LOC to any referential solution.

\subsection{Task requirements}

The TO-DO application's UI will consist of multiple elements.
The \emph{InputField} is a text input element describing a new task.
\emph{AddTodo} is a button that creates a new task.
The \emph{Todo} displays a checkbox and the description of a particular task. The checkbox represents if the Todo is completed.
\emph{Todos} displays the created Todo elements.
\emph{CompletedNum} displays the number of completed tasks.
Moreover, lastly, the \emph{AllDoneButton} is used to mark all Todo elements as completed.

The functionality we will try to implement is the following:
\begin{enumerate}
	\item The user can input text into the InputField element.
	\item When the user clicks the AddTodo and the InputField is not empty, create a new Todo element and display it in the Todos.
	\item The CompletedNum reflects how many Todo elements have their checkbox checked.
	\item After clicking the AllDone button, all Todo elements become completed.
\end{enumerate}
\medskip
\subsection{Creation process}

The steps of the creation process are the following:
\begin{enumerate}
\item \textbf{Prepare JSON data:} The creation process starts by creating JSON data based on the format specified by the task.
We can see the created JSON object in Figure~\ref{fig:todo-json}.
We model the data based on the UI elements we wish to create according to our specification.
\begin{figure}[htbp]
	\caption{JSON object created as input for the TO-DO list task}
	\label{fig:todo-json}
	\begin{lstlisting}
{
  "InputField": "",
  "AddTodo": "Add todo",
  "Todos": [
      {
          "text": "Complete project proposal",
          "completed": false
      }
  ],
  "Others": {
      "CompletedCount": 0,
      "AllDoneButton": "All done"
  }
}
    \end{lstlisting}
\end{figure}


\item \textbf{Upload data to the system:} We upload the created JSON data to the system.
\item \textbf{Replace all hole elements:} We use the provided context menus to replace the hole elements with the new UI elements mirroring the data's structure.
\item \textbf{Change the order of the elements:} Using the top-level KeyOrdering menu we change the order of the elements.
\item \textbf{Change the tags of the elements:} We use the provided context menus to change the tags for the elements.
We select the \emph{input} tag for the InputField and Todo.Completed elements. We choose the \emph{button} tag for the Other.AllDoneButton and AddTodo elements.
After that, we choose to make the Todos ordered.
\item \textbf{Add necessary attributes:}
\begin{itemize}
	\item \emph{Completed element:} We add the \emph{type} attribute to the \emph{Completed} element, select the value as \emph{Constant} and input \emph{checkbox}.
	      Then we add the checked attribute and select the \emph{Data} InnerValue.
	\item \emph{InputField element:} We add the \emph{type} attribute with the InnerValue \emph{Constant} set to \emph{text}.
	      After that, we add the \emph{value} attribute and select the \emph{Data} InnerValue.
\end{itemize}
\item \textbf{Create custom Messages:} To implement custom functionality, we use the provided canvas menu for creating messages to create 4 messages which modify the state of the application based on UI events.
The application automatically creates the corresponding update function cases for each message.
\begin{itemize}
\item \textbf{UpdateInput:}  Using the canvas menu, we type the following JavaScript code into the editor window to create the UpdateInput message which we can see in Program~\ref{fig:todo-updateInput}.

\begin{listing}[htbp]
\caption{Update function case for the UpdateInput message.}
\label{fig:todo-updateInput}
\begin{lstlisting}
return {
  ...model,
  InputField: event.target.value
};
            \end{lstlisting}
\end {listing}


\item \textbf{AddTodo:} Using the canvas menu, we type the following JavaScript code into the editor window to create the AddTodo message which we see in Program~\ref{fig:todo-AddTodo}.

\begin{listing}[htbp]
	\caption{Update function case for the AddTodo message.}
	\label{fig:todo-AddTodo}
	\begin{lstlisting}
if (!model.InputField.trim()) return model;
return {
  ...model,
  InputField: "",
  Todos: [...model.Todos, {
      text: model.InputField.trim(),
      completed: false
  }]
};
            \end{lstlisting}
\end{listing}

\item \textbf{ToggleTodo:} Using the canvas menu, we type the following JavaScript code into the editor window to create the ToggleTodo message which we see in Program~\ref{fig:todo-ToggleTodo}.
\begin{listing}[htbp]
	\caption{Update function case for the ToggleTodo message.}
	\label{fig:todo-ToggleTodo}
	\begin{lstlisting}
const todoIndex = 
    parseInt(event.target.closest('li').dataset.index);
const updatedTodos = model.Todos.map((todo, index) =>
  index === todoIndex
    ? {...todo, completed: !todo.completed}
    : todo
);
return {
  ...model,
  Todos: updatedTodos,
  Others: {
    ...model.Others,
    CompletedCount: updatedTodos.filter(todo => todo.completed).length
  }
};
            \end{lstlisting}
\end{listing}

\item \textbf{CompleteAll:} Using the canvas menu, we type the following JavaScript code into the editor window to create the CompleteAll message which we see in Program~\ref{fig:todo-CompleteAll}.


\begin{listing}[htbp]
	\caption{Update function case for the CompleteAll message.}
	\label{fig:todo-CompleteAll}
	\begin{lstlisting}
const allCompleted = model.Todos.map(todo => 
  ({...todo, completed: true}));
  return {
    ...model,
    Todos: allCompleted,
    Others: {
      ...model.Others,
      CompletedCount: allCompleted.length
    }
  };
  \end{lstlisting}
\end{listing}

\end{itemize}

\item \textbf{Attach custom handlers to elements:} Using the provided EventHandler menus, we add 4 EventHandlers to 4 of the created elements:
\begin{itemize}
	\item We add the \emph{onChange} event with the message handler \emph{UpdateInput} to the InputField element.
	\item We add the \emph{onClick} event with the message handler \emph{AddTodo} to the AddTodo element.
	\item We add the \emph{onChange} event with the message handler \emph{ToggleTodo} to the Todo.completed element.
	\item We add the \emph{onClick} event with the message handler \emph{CompleteAll} to the Others.AllDoneButton element.
\end{itemize}
\end{enumerate}
\medskip
\subsection{Results}
We \textbf{successfully} created the TO-DO list application and implemented its functionality and UI elements..
As we already described the individual update cases based on defined messages, we see the remaining parts generated by the \emph{InterfaceSmith} in Program~\ref{fig:todo-result},
except the \emph{render} and \emph{init} functions common to all applications generated by our system.

\begin{listing}[p]
	\caption {The TODO list implementation generated by the \emph{InterfaceSmith} based on the interactions with the system(update, render and init functions not included).}
	\label{fig:todo-result}
	\begin{lstlisting}
const Msg = {
  UpdateInput: "UpdateInput",
  AddTodo: "AddTodo",
  ToggleTodo: "ToggleTodo",
  CompleteAll: "CompleteAll",
};

const Model = {
  InputField: "",
  AddTodo: "Add todo",
  Todos: [
    {
      text: "Complete project proposal",
      completed: false,
    },
  ],
  Others: {
    CompletedCount: 0,
    AllDoneButton: "All done",
  },
};

const view = (
  model,
) => `<div  >
  <input value="${model.InputField}" type="text" onBlur="window.dispatch(Msg.UpdateInput, event)" />
  <button  onClick="window.dispatch(Msg.AddTodo, event)">${model.AddTodo}</button>

<ul  >${model.Todos.map(
  (item, index) => `
  <li data-index="${index}"><div  ><input type="checkbox" ${item.completed ? "checked" : ""} 
    onChange="window.dispatch(Msg.ToggleTodo, event)" />
  <span  >${item.text}</span></div></li>`,
).join("")}</ul>

<div  ><button  onClick="window.dispatch(Msg.CompleteAll, event)">${model.Others.AllDoneButton}</button>
<div  >${model.Others.CompletedCount}</div></div></div>`;
\end{lstlisting}
\end{listing}

\medskip
\subsection{Analysis}
As this particular task implements only a subset of the functionality defined by \citet{todo-spec}, we only analyze the number of LOCs needed to implement the desired functionality.
We now analyze the Number of LOCs needed to implement the Counter task using our programming system:
\begin{itemize}
	\item In Step 1, we created the input data and wrote 14 LOC.
	\item   In Steps 2-5, we created and modified the UI elements using only mouse-based operations, resulting in 0 LOC written.
	\item   In Step 6, we added four attributes in total to two different UI elements using the provided modification menus, and we needed to specify Constant InnerValues for 3 of them, resulting in 3 LOC written.
	\item   In Step 7, we created the custom functionality by writing JavaScript code, resulting in 37 LOC written.
	\item In Step 8, we added the custom handlers to four UI elements using only mouse-based operations, resulting in 0 LOC written.
\end{itemize}
\noindent We wrote \textbf{54} lines of physical code in total, and the rest of the operations were performed through a mouse-based interface.



\clearpage
\section{Counter Task (7GUIs)}
The Counter task is defined by the \citet{7GUIs-web} as a straightforward task meant to introduce the technology used, its features, and required scaffolding.
It consists of two UI elements: a button and a current count label. When the button is clicked, the counter's value is incremented by one.

\subsection{Task requirements}
\subsubsection{UI elements}
The main UI elements of the application are:
\begin{enumerate}
	\item Count: A label displaying the current value of the \emph{Count} field.
	\item incrementButton: A button used to increment the Count.
\end{enumerate}

\subsubsection{Functionality}
The functionality we will try to implement is the following:
\begin{enumerate}
	\item The user can click on the provided button, and the Count is incremented by 1. The Count element is updated to reflect the change in value.
\end{enumerate}
\medskip
\subsection {Creation process}

The steps of the creation process are the following:
\begin{enumerate}
	\item \textbf{Prepare JSON data:} The creation process starts by creating JSON data based on the format specified by the task.
	      We can see the created JSON object in Figure~\ref{fig:counter-json}.
	\item \textbf{Upload data to the system:} We upload the created JSON data to the system.
	\item \textbf{Replace holes with new elements:} This step involves clicking on a provided button menu for each Hole element.
	\item \textbf{Modify the elements using the context menus:} We use the provided context menus to change the tag of each element.
	      We select the label tag for the Count element and the button for the incrementButton element.
	\item \textbf{Implementation of custom behavior:} Using the canvas menu, we create the Increment message, specify its name, and type the following JavaScript code into the editor window to implement the message:
	      \begin{listing}[htbp]
		      \caption{Update function case for the Increment message.}
		      \begin{lstlisting}
return {
  ...model, Count: model.Count + 1
};
            \end{lstlisting}
	      \end{listing}
	\item \textbf{Add the EventHandler to the button element:} We add the \emph{onClick} event with the message handler \emph{Increment} to the incrementButton element.
\end{enumerate}


\begin{figure}[htbp]
	\caption{JSON object created as input for the Counter task (7GUIs)}
	\label{fig:counter-json}
	\begin{lstlisting}
{
  "Count": 0,
  "incrementButton": "Increment counter"
 }
    \end{lstlisting}
\end{figure}
\medskip
\subsection{Results}
We \textbf{successfully} created the desired application and implemented its functionality and UI elements.
We can see the code generated by the \emph{InterfaceSmith} system in Program~\ref{fig:counter-result}.
\medskip
\subsection{Analysis}
The referential 7GUIs solution~\cite{7guis-React-TypeScript-MobX/src/app/guis/counter.tsx} for the counter task consists of 11 LOCs written in TypeScript.


\begin{listing}[p]
	\caption {The full Counter task implementation generated by the \emph{InterfaceSmith} system.}
	\label{fig:counter-result}
	\begin{lstlisting}
const Msg = {
  Increment: "Increment",
};

const Model = {
  Count: 0,
  incrementButton: "Increment counter",
};

const update = (msg, event, model) => {
  switch (msg) {
    case Msg.Increment:
      return {
        ...model,
        Count: model.Count + 1,
      };

    default:
      return model;
  }
};

const view = (model, dispatch) => `
<div  >
<label  >${model.Count}</label>
<button  onClick="window.dispatch(Msg.Increment, event)">${model.incrementButton}</button>
</div>`;

function startApplication(initialModel, updateFunction, viewFunction) {
  let currentModel = initialModel;
  const render = () => {
    const root = document.getElementById("app");
    root.innerHTML = viewFunction(currentModel, dispatch);
  };
  window.dispatch = (msg, event) => {
    currentModel = updateFunction(msg, event, currentModel);
    render();
  };
  render();
}

startApplication(model, update, view);
\end{lstlisting}
\end{listing}


We will now analyze the Number of LOCs needed to implement the Counter task using our programming system
\begin{itemize}
	\item In Step 1, we created the input data and wrote a total of 4 LOC.
	\item   In Steps 2-4, we created and modified the UI elements using only mouse-based operations, resulting in 0 LOC written.
	\item   In Step 5, we created a custom messahe, for which we wrote its name and implementation, resulting in 4 LOC written.
	\item   In Step 6, we added an EventHandler to the \texttt{button} element through mouse-based operations, resulting in 0 LOC written.
\end{itemize}
\noindent We wrote \textbf{8} lines of physical code in total, and the rest of the operations were performed through a mouse-based interface.







\clearpage
\section{Temperature Converter Task (7GUIs)}
Temperature Converter task is defined by the \citet{7GUIs-web} as a more complicated task, as it requires implementing a \emph{bidirectional data flow} between
two input elements, and also requires implementing custom validation of user input.

\medskip
\subsection{Task requirements}
\subsubsection{UI elements}
The main UI elements of the application are:
\begin{enumerate}
	\item Celsius: An input element displaying the current value of the \emph{Celsius} field.
	\item Fahrenheit: A input element displaying the current value of the \emph{Fahrenheit} field
\end{enumerate}

\subsubsection{Functionality}
The functionality we will try to implement is the following:
\begin{enumerate}
	\item The user can change the value of either the Celsius or Fahrenheit elements, and it automatically converts the value of one to the other and updates the elements.
	\item If the user inputs a non-numerical value, do not update the other field's value.
\end{enumerate}
\medskip
\subsection {Creation process}
The steps of the creation process are the following:
\begin{enumerate}
	\item \textbf{Prepare JSON data:} The creation process starts by creating JSON data based on the format specified by the task.
	      We can see the created JSON object in Figure~\ref{fig:temp-json}.
	      \begin{figure}[H]
		      \caption{JSON object created as input for the Temperature Converter Task (7GUIs)}
		      \centering
		      \label{fig:temp-json}
		      \begin{lstlisting}
{
  "Celsius": "",
  "CelsiusLabel": "Celsius = ",
  "Fahrenheit": "",
  "FahrenheitLabel": "Fahrenheit"
}
    \end{lstlisting}
	      \end{figure}
	\item \textbf{Upload data to the system:} We upload the created JSON data to the system.
	\item \textbf{Replace holes with new elements:} This step involves clicking on a provided button menu for each Hole element.
	\item \textbf{Modify the elements using the context menus:} We use the provided context menus to change the tag of each element.
	      We select the label tag for the CelsiusLabel and FahrenheitLabel elements and the input tag for the Celsius and Fahrenheit elements.
	\item \textbf{Implementation of custom behavior:} The implementation of the custom messages is inspired by the~referential~solution~\cite{7guis-React-TypeScript-MobX/src/app/guis/tempconv.tsx}. We define 2 new messages and modify their names.
	      \begin{itemize}
		      \item UpdateCelsius update function implementation:
		            \begin{listing}[htbp]
			            \caption{Update function case for the UpdateCelsius message.}
			            \begin{lstlisting}
if (isNaN(parseFloat(event.target.value))) {
  return {
    ...model,
    Celsius: event.target.value
  };
}
let celsius = parseFloat(event.target.value);
let fahrenheit = (celsius * 9 / 5 + 32).toFixed(1);
return {
  ...model,
  Celsius: celsius,
  Fahrenheit: fahrenheit.toString()
};            \end{lstlisting}
		            \end{listing}

		      \item UpdateFahrenheit update function implementation:
		            \begin{listing}[htbp]
			            \caption{Update function case for the UpdateFahrenheit message.}

			            \begin{lstlisting}
if (isNaN(parseFloat(event.target.value))) {
  return {
    ...model,
    Fahrenheit: event.target.value
  };
}
let fahrenheit2 = parseFloat(event.target.value);
let celsius2 = ((fahrenheit2 - 32) * 5 / 9).toFixed(1);
return {
  ...model,
  Fahrenheit: fahrenheit2,
  Celsius: celsius2.toString()
};            \end{lstlisting}

		            \end{listing}
	      \end{itemize}

	\item \textbf{Add the EventHandlers to the input elements:} We add the \emph{onChange} event with the message handler \emph{UpdateCelsius} to the Celsius element and the \emph{onChange} event with the message handler \emph{UpdateFahrenheit} to the Fahrenheit element.
\end{enumerate}
\medskip
\subsection{Results}
We \emph{successfully} created the desired application and implemented its functionality and UI elements.
We can see the code generated by the \emph{InterfaceSmith} system in Program~\ref{fig:converter-result}, not including the already described update function.

\begin{listing}[p]
	\caption {The Counter task implementation generated by the \emph{InterfaceSmith} system(update function not included).}
	\label{fig:converter-result}
	\begin{lstlisting}
const Msg = {
  UpdateCelsius: "UpdateCelsius",
  UpdateFahrenheit: "UpdateFahrenheit",
};

const Model = {
  Celsius: "",
  CelsiusLabel: "Celsius = ",
  Fahrenheit: "",
  FahrenheitLabel: "Fahrenheit",
};

const view = (
  model,
  dispatch,
) =>
`<div  ><input value="${model.Celsius}" onChange="window.dispatch(Msg.UpdateCelsius, event)" />
<label  >${model.CelsiusLabel}</label>
<input value="${model.Fahrenheit}" onChange="window.dispatch(Msg.UpdateFahrenheit, event)" />
<label  >${model.FahrenheitLabel}</label></div>`;

function startApplication(initialModel, updateFunction, viewFunction) {
  let currentModel = initialModel;
  const render = () => {
    const root = document.getElementById("app");
    root.innerHTML = viewFunction(currentModel, dispatch);
  };
  window.dispatch = (msg, event) => {
    currentModel = updateFunction(msg, event, currentModel);
    render();
  };
  render();
}

startApplication(model, update, view);
\end{lstlisting}
\end{listing}


\medskip
\subsection{Analysis}
The referential 7GUIs solution~\cite{7guis-React-TypeScript-MobX/src/app/guis/tempconv.tsx} for the Temperature Converter task consists of 66 LOCs written in TypeScript using the \citet{react} library.

We will now analyze the Number of LOCs needed to implement the Counter task using our programming system
\begin{itemize}
	\item In Step 1, we created the input data and wrote a total of 6 LOC.
	\item   In Steps 2-4, we created and modified the UI elements using only mouse-based operations, resulting in 0 LOC written.
	\item   In Step 5, we created a custom function, for which we wrote its name and implementation, resulting in 28 LOC written.
	\item   In Step 6, we added the EventHandlers to both input elements through mouse-based operations, resulting in 0 LOC written.
\end{itemize}
\noindent We wrote \textbf{34} lines of physical code in total, and the rest of the operations were performed through a mouse-based interface.



\clearpage
\section{Evaluation}
In this section, we summarize our findings from the benchmark tasks.
Our benchmarking process involved implementing three tasks using the \emph{InterfaceSmith} low-code programming system.
Our main considerations for this benchmarking are whether we can use our system to implement the specific tasks and how many lines of code we need to write to achieve the desired functionality.

We successfully implemented all three tasks using our programming system, which were the \emph{Counter task}\cite{7GUIs-web},
the \emph{Temperature Converter task}\cite{7GUIs-web}, and a simple \emph{TO-DO list} application inspired by the \citet{TodoMVC} project.
The implementation of each task consisted of obtaining JSON data of a particular structure, uploading the data to the system, creating the UI using the system's low-code context menus, and implementing custom functionality by
creating new \emph{message} handlers, which we attached to specific elements triggered by certain events.

The \emph{Counter} task involved creating a simple JSON input, creating the UI using our system's low-code interface, and then implementing a simple update message,
which is passed to the update function when the button is clicked, and the state is updated.
The task demonstrated that we can create a simple web application using our system, which supports state management and user interaction reactivity.

The \emph{Temperature Converter} task involved creating a simple JSON input, creating the two input fields for the Celsius and Fahrenheit temperatures, and then implementing the desired update functionality based on the user
interactions. As explained by \citet{7GUIs-web}, this task demonstrated that our system provides the necessary tools to implement \emph{bidirectional data flow} between the two input fields.

The \emph{TO-DO list} was the most complex of the selected tasks. It involved creating a JSON input, creating all necessary UI elements, and then implementing the functionality by defining four messages and their implementation.
The task demonstrated that we can create more complex applications using the \emph{InterfaceSmith} programming system.

Other than successfully implementing the functionality, our second main consideration is the number of lines of code needed to implement the tasks using our system.
Table~\ref{tab:results} displays the results of the benchmarking process regarding the amount of code we needed to write to implement the custom functionality and UI.
Each row of the table corresponds to one of the three benchmarking tasks.
The numerical value in the first column indicates the total number of lines of code needed to fully implement the corresponding task.
In the second column, the value shows how many lines of code we needed to write in advance to create the input data.
The third column shows the number of lines of code needed to implement the specific custom functionality.
The third column shows the number of lines of code for each task's referential solution.
The fifth column shows whether we successfully implemented the task according to its specification.


\begin{table}[htbp]
	\centering
	\begin{tabular}{|p{3cm}|c|c|c|c|c|}
		\hline
		\textbf{Task}   & \textbf{Total} & \textbf{Prep} & \textbf{Custom} & \textbf{Ref.} & \textbf{Success} \\
		\hline
		TO-DO List      & 54             & 14            & 40              & N/A           & Yes              \\
		Counter         & 8              & 4             & 4               & 11            & Yes              \\
		Temp. Converter & 34             & 6             & 28              & 66            & Yes              \\
		\hline
	\end{tabular}
	\caption{Implementation Results Summary}
	\label{tab:results}
\end{table}


The results for the \emph{TO-DO list} task show that we needed to write 54 lines of code in total, where we wrote 14 LOC to prepare the data and 40 LOC to implement the custom functionality.
The results for the \emph{Counter} task, displayed on the second row, show that we needed to write 8 LOC in total, 4 of which we wrote to prepare the data and 4 LOC to implement the functionality.
For the \emph{Temperature Converter} task, we needed to write 34 LOC in total, and we needed to write 6 LOC to prepare the data and 28 LOC to implement the custom behavior.

Before we compare the amount of LOC needed for each task to their referential solutions, we must state that the referential solutions are implemented in \emph{TypeScript} using the \citet{react} library,
which is different from the pure JavaScript our system generates.
However, this comparison is intentional, as we want to compare the amount of work needed to implement the tasks using our system to the typical purely text-based implementation approach.
The referential solution for the \emph{Counter} tasks is 11 LOC long compared to 8 LOC needed to implement it using our system.
The \emph{Temperature Converter} task's referential solution consists of 66 LOC, whereas using our programming system, we only needed to write 34 LOC to implement the same functionality.


Table~\ref{tab:results-percentages} illustrates the code distribution between data preparation and custom functionality implementation across all tasks.
Similarly to Table~\ref{tab:results}, each row shows code distribution for one of the tasks.
In the first column, we can see the total LOC needed to implement the tasks.
In the second column, we can see the percentage of lines of code we wrote to prepare the data compared to the total number of lines of code,
The last column shows the percentage of code needed to implement the custom functionality compared to the total number of lines of code.
We see that for the \emph{TO-DO list} and \emph{Temperature Converter} tasks, we needed to write considerably more code to implement the custom functionality than the lines of code needed to create the input data.

\begin{table}[htpb]
	\centering
	\begin{tabular}{|p{3cm}|c|c|c|}
		\hline
		\textbf{Task}   & \textbf{Total LOC} & \textbf{Data Prep (\%)} & \textbf{Custom Logic (\%)} \\
		\hline
		TO-DO List      & 54                 & 26\%                    & 74\%                       \\
		Counter         & 8                  & 50\%                    & 50\%                       \\
		Temp. Converter & 34                 & 18\%                    & 82\%                       \\
		\hline
	\end{tabular}
	\caption{Implementation Results with Code Distribution Analysis}
	\label{tab:results-percentages}
\end{table}

