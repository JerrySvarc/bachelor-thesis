\chapter{Benchmarks}
\label{chap:walktrough}

In this chapter, we evaluate our Data-driven UI prototype system on several tasks. Our primary goal is to assess whether we can implement these tasks according to their specifications using only our prototype system, and to determine if our approach successfully reduces the amount of code that needs to be written, as per the definition of low-code programming systems by \citet{Pinho_Aguiar_Amaral_2023}.

\section{Methodology}
We will evaluate our prototype system based on the following criteria:
\begin{itemize}
	\item \textbf{Successful implementation of all specified functionality:}
	      We must implement the exact functionality described by each task.
	      If we cannot implement a specific task as specified, this allows us to reason about our system's limitations, and potentially
	      identify problems with our implementation or design.

	\item \textbf{Number of lines of code written:}
	      We will compare the number of lines of code required in our system versus the referential provided solution.
	      We will only consider \emph{physical lines of code} as defined by~\citet{Park_1992}.
	      Because our system requires concrete data before we can begin building the desired application, we will also include the
	      number of lines of code needed to be written for the data preparation.
	      We will also ignore all lines of code ralated to styling the elements and focus only on implementing the specified functionality and UI structure.

\end{itemize}

For each specific task, we will:
\begin{enumerate}
	\item Describe the task requirements
	\item Outline the implementation process using our Data-driven UI system
	\item Present the resulting application or describe what functionality cannot be implemented
	\item Analyze the number of lines of code written
	\item Discuss any challenges or limitations encountered
\end{enumerate}

\section{Counter Task (7GUIs)}
The Counter is defined by \citet{7GUIs-web} as follows: ``The task is to build a frame containing a label or read-only textfield T and a button B. Initially, the value in T is “0” and each click of B increases the value in T by one.''

\subsection {Creation process}

The steps of the creation process are the following:
\begin{enumerate}
	\item \textbf{Prepare JSON data:} The creation process starts by creating JSON data based on the format specified by the task.
	      We can see the created JSON object in Figure~\ref{fig:counter-json}.
	\item \textbf{Upload data to the system:} We upload the created JSON data to the system.
	\item \textbf{Replace holes with new elements:} This step involves clicking on a provided button menu for each Hole element.
	\item \textbf{Modify the elements using the context menus:} We use the provided context menus to change the tag of each element.
	      We select the label tag for the label element, and button for the button element.
	\item \textbf{Implementation of custom behavior:} We select the \emph{Custom Handlers} menu.
	      We type the name of our custom function into the provided input field and
	      we name the function as \texttt{plusOne}.
	      Then using JavaScript, we implement the desired behavior of increasing the value of the label field by one
	      every time this function is called.
	      Into the editor window we input the following: \texttt{data.label++;}.
	      We do not need to write anything else, as the system automatically generates the function's body and other event handlers.
	\item \textbf{Add the "onClick" handler to the button element:} We use the menu for the \emph{button} element to add an event handler
	      which calls the \texttt{plusOne} function when triggered.
\end{enumerate}


\begin{listing}[htbp]
	\caption{JSON object created as input for the Counter task (7GUIs)}
	\label{fig:counter-json}
	\begin{lstlisting}
{
  "label": 0,
  "button": "count"
 }
    \end{lstlisting}
\end{listing}

\subsection{Results}
We have successfully created the desired application and its specified functionality and UI elements.
We can observe the working application in the code generated by the prototype system in Appendix~\ref{fig:counter-res}.
We can see the HTML and JavaScript code



\subsection{Analysis}


\section{Temperature Converter Task (7GUIs)}
 [Follow the same structure as the Counter task]

\section{To-do Application}
 [Describe the requirements for a simple to-do application]
 [Outline the steps taken to implement it in Data-driven UI]
 [Show screenshots or code snippets of the implementation]
 [Analyze the lines of code written]
 [Discuss any challenges or limitations]

\section{Evaluation}
In this section, we'll summarize our findings from the benchmark tasks:
[Discuss overall success in implementing the tasks]
[Compare lines of code written in our system vs traditional approaches]
[Highlight strengths of the Data-driven UI approach]
[Address any common limitations or challenges encountered]
[Reflect on how well the system meets the definition of a low-code programming system]
