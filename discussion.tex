\chapter{Discussion}
\label{chap:discussion}

Every approach to software development has associated benefits and limitations.
For example, \citet{Pinho_Aguiar_Amaral_2023} identified several benefits and limitations of existing Low-code programming systems.
Some of the common benefits are the following:
\begin{itemize}
	\item \textbf{Low requirements for technical skills}
	\item \textbf{High speed/short development time}
	\item \textbf{Allows for learning concepts}
\end{itemize}



In previous Chapters, we explored the design concepts and implementation of our proposed \emph{Data-driven UI} programming system.
We also tested the prototype implementation on various simple tasks, such as a simple \emph{To-do application}.
In this Chapter, we will critically evaluate our work and describe its benefits and limitations, as well as potential avenues of future research.


\section{Benefits of the approach}

We will focus on the benefits of the Data-driven Low-code programming approach.

\section{Limitations of the approach}

\section{Limitations of the implementation}

\section{Future work}

