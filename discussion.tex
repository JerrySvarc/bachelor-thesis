\chapter{Discussion}
\label{chap:discussion}

Every approach to software development has associated benefits and limitations.
For example, \citet{Pinho_Aguiar_Amaral_2023} identified several benefits and pitfalls associated with existing Low-code programming systems.
Some of the common benefits and limitations are the following:
\begin{itemize}
	\item Benefits:
	      \begin{itemize}
		      \item \textbf{Low requirements for technical skills}
		      \item \textbf{High speed/short development time}
		      \item \textbf{Allows for learning concepts}
	      \end{itemize}

	\item Pitfalls:
	      \begin{itemize}
		      \item \textbf{Interoperability issues}
		      \item \textbf{Vendor lock-in}
		      \item \textbf{Writing more code than intended}
	      \end{itemize}

\end{itemize}


In this chapter, we will critically evaluate our work and describe its benefits, limitations, and potential avenues for future research.
We will separate the limitations of the Data-driven Low-code approach from the limitations of our Data-driven UI programming system prototype as to
better evaluate the potential feasibility of the approach itself.

\section{Benefits of the approach}

We will focus on the benefits of the Data-driven Low-code programming approach.

\section{Limitations of the approach}

\section{Limitations of the implementation}

\section{Future work}

