\chapter{Discussion}
\label{chap:discussion}

Every approach to software development has associated benefits and limitations.
For example, \citet{Pinho_Aguiar_Amaral_2023} identified several benefits and pitfalls associated with existing Low-code programming systems.
Some of the common benefits and limitations of existing Low-code programming systems are the following:
\begin{itemize}
	\item Common Benefits:
	      \begin{itemize}
		      \item \emph{Low requirements for technical skills}
		      \item \emph{High speed/short development time}
		      \item \emph{Allows for learning concepts}
	      \end{itemize}

	\item Common Pitfalls:
	      \begin{itemize}
		      \item \emph{Interoperability issues}
		      \item \emph{Vendor lock-in}
		      \item \emph{Writing more code than intended}
	      \end{itemize}
\end{itemize}


In this chapter, we will critically evaluate our work and describe its benefits, limitations, and potential avenues for future research.
We will separate the limitations of the Data-driven Low-code approach from the limitations of our Data-driven UI programming system prototype as to
better evaluate the potential feasibility of the approach itself.

\section{Benefits of the approach}

In our work, we identify several benefits of the Data-driven Low-code approach in the context of creating client-side web applications.
Each benefit is a direct result of our specific design principles as described in Chapter~\ref{chap:design}.
The benefits are the following:
\begin{itemize}
	\item \textbf{Guided UI element creation:} The Data-driven Hole-based approach allows for a guided process of creating the desired UI elements.
	      The input data inspires the created elements and their resulting structure.
	      This process relieves the cognitive load associated with creating UI elements from scratch, as the system automatically creates the right type
	      of element and places it at the correct place in the internal representation of UI elements.
	\item \textbf{Mitigates Vendor lock-in:}
	\item \textbf{Mitigates Interoperability issues:}
\end{itemize}


\section{Limitations of the approach}

\begin{itemize}
	\item \textbf{Necessity to prepare data in advance:}
	\item \textbf{Potentially resource intensive:}
	\item \textbf{Low-code interface design:}
\end{itemize}

\section{Limitations of the implementation}

\section{Future work}

