\chapter{Core logic}
\label{chap:corelogic}

This chapter describes the core logic of the programming system.
Firstly, we describe the hole-based approach applied to the creation of elements.
After that, we describe the type system, mainly the internal representation of the UI elements and necessary operations on these types.
Then, we describe mapping the input data types to specific types representing the UI elements.
We present two ways of combining the created UI elements with the corresponding data.
Lastly, we describe one possible way of displaying a preview of the created UI elements and the modification menus.

\section{Hole-based approach}
\label{sec:hole-based}
The primary motivation of this approach is to allow \emph{incremental creation} of UI elements.
The elements are created based on the provided data and its structure.
Each element corresponds to a data value.
We need the ability to represent data values that have yet to be used to create UI elements.
To address this need, we define a type named \emph{Hole} as follows:
\begin{defn}[Hole type definition]
	A Hole is a placeholder type representing a UI element that is yet to be created despite its existing corresponding data. It serves as a temporary stand-in during the incremental construction of a user interface.
\end{defn}

We then define the incremental creation process as a \emph{sequence of discrete opreations}.
Each operation is either a \emph{modification} of an existing UI element or a \emph{replacement} of a Hole element with a new UI element based on the corresponding data.
This discrete approach allows the system to perform different tasks after each operation.
These tasks include updating the UI element preview, analyzing the input data, or providing new modification menus and options.
The system could also analyze the existing UI elements and other operations.


\section{Type system and Operations}

\subsection{JSON type}
\label{sub:json}
The creation process starts by providing \emph{JSON} data to the system.
In order to use this data, we parse it and create an internal representation.
We define a discriminated union type named \emph{JSON}.
This type is the foundation for the Abstract Syntax Tree (AST) representation of the JSON data.
The type is defined as a as follows:
\begin{listing}[H]
	\caption {JSON type}
	\begin{lstlisting}
  type Json =
    | JNumber of float
    | JString of string
    | JBool of bool
    | JNull
    | JArray of Json list
    | JObject of Map<string, Json>
  \end{lstlisting}
\end{listing}

Each node represents a JSON value in the input data.
The types of nodes can be divided into two categories:
\begin{itemize}
	\item {\textbf{Collections:} The first category contains types representing a \emph{collection} of other values. This category includes the types \emph{JObject} and \emph{JArray}.
	      We define the two collection types as follows:
	      \begin{itemize}
		      \item \textbf{JObject:} It is based on the JSON Object type and represents a collection of different JSON types. The original ordering of the inner elements is \emph{ignored}.
		      \item \textbf{JArray:} It is based on the JSON List type and represents a collection of JSON values of the \emph{same type and structure}. The original ordering of the inner elements is \emph{preserved}.
	      \end{itemize}
	      }
	\item \textbf{Primitives:} The second category contains types representing data primitives such as numerical values, boolean values,
	      a string of text, or the null value.
\end{itemize}

\subsection{RenderingCode type}
To facilitate the creation of UI elements based on the JSON type, we define a type named \emph{RenderingCode}.
The \emph{RenderingCode} is a discriminated union type used to represent the UI elements.
Similarly to the JSON type, each case represents a type of an HTML element with a corresponding mapping to the JSON type.
This type is the foundation for the \emph{Abstract Syntax Tree} representation of UI elements.
\begin{listing}[H]
	\caption {RenderingCode type}
	\begin{lstlisting}
  type RenderingCode =
    | HtmlElement of
        tag: Tag *
        attrs: Attributes *
        innerValue: InnerValue
    | HtmlList of
        listType: ListType *
        elements: RenderingCode list
    | HtmlObject of
      objectType : ObjectType *
      elements: RenderingCode list
    | Hole
  \end{lstlisting}
\end{listing}

The case types are defined as follows:
\begin{itemize}
	\item \textbf{HtmlElement:} Represents a single HTML element. It consists of:
	      \begin{itemize}
		      \item \emph{Tag}: Denotes the HTML tag (e.g., div, p, span).

		      \item \emph{InnerValue}: Represents the content of the element, defined as:
		            \begin{lstlisting}
type InnerValue =
  | Data 
  | Empty
  | Constant of string  
        \end{lstlisting}
		            Where \emph{Data} indicates dynamic content from JSON, \emph{Empty} represents an empty element, and \emph{Constant} holds static, developer-defined content.
	      \end{itemize}	\item \textbf{HtmlList:} Represents a collection of similar HTML elements corresponding to JSON arrays. It includes:
	      \begin{itemize}
		      \item \emph{ListType}: Specifies the list type (e.g., unordered, ordered, table).
		      \item \emph{itemCodes}: A list of RenderingCode elements representing list items.
		      \item {\emph{Attributes}: A collection of key-value pairs representing HTML attributes.
		            The Attribute type is defined as follows:
		            \begin{lstlisting}
type Attribute = string * InnerValue
              \end{lstlisting}
		            It consists of a key of a type string and an value of a type InnerValue. This allows selecting the value of an Attribute as a value from the input data.
		            }

	      \end{itemize}
	\item \textbf{HtmlObject:} Represents a collection of diverse HTML elements derived from JSON object. It consists of:
	      \begin{itemize}
		      \item \emph{ObjectType}: Defines the type of the object (e.g., div, span, section).
		      \item \emph{Elements}: A list of RenderingCode elements representing the object's contents.
	      \end{itemize}
	\item {\textbf{Hole:} Represents a placeholder in the UI element structure as defined in Section~\ref{sec:hole-based}, allowing for dynamic element creation.
	      The Hole is a discriminated union type
	      }


	\item  \xxx{//TODO: CustomWrapper and Javascript}
\end{itemize}


\subsection{Data Mapping: JSON to RenderingCode}
The process of transforming JSON data into UI elements requires systematic mapping between JSON and RenderingCode types.
This mapping forms the core of our UI generation system, allowing us to convert arbitrary JSON input into a structured representation of UI elements.
Each JSON type has a corresponding representation in RenderingCode:
\begin{itemize}
	\item JObject maps to HtmlObject
	\item JArray maps to HtmlList
	\item JNull, JString, JNumber, and JBool map to HtmlElement
\end{itemize}
Th

We define a mapping function named \emph{recognizeJson} as follows:
\begin{listing}[H]
	\caption {RenderingCode type}
	\begin{lstlisting}
let recognizeJson (json: Json) =
    match json with
    | JArray array -> 
        //create a hole for each element of the array
        HtmlList containing the holes as elements
    | JObject obj ->
        //create hole for each element of the object 
        HtmlObject containing the holes as elements
    | JNull -> Hole(UnNamed)
    | _ -> HtmlElement 
  \end{lstlisting}
\end{listing}


\section{Creation process}



\section{Referencing input data}
\subsection{Structual referencing}
\subsection{Selector based approach}

