\chapter{Core logic}
\label{chap:corelogic}

This chapter describes the core logic of the programming system.
Firstly, we describe the hole-based approach applied to the creation of elements.
After that, we describe the type system, mainly the internal representation of the UI elements and necessary operations on these types.
Then, we describe mapping of the different input data types to specific types representing the UI elements.
We present two ways of combining the created UI elements with the corresponding data.
Lastly, we describe one possible way of displaying a preview of the created UI elements and the modification menus.

\section{Hole-based approach}
The hole−based approach is

\section{Type system}
\subsection{RenderingCode}
The \emph{RenderingCode} is a discriminated union type used to represent the UI elements.
Each case represents a type of an HTML element.


\begin{listing}[H]
	\label{list:holeCode}
	\caption {RenderingCode type with hole}
	\begin{lstlisting}
  type RenderingCode =
    | HtmlElement of 
        tag: Tag * 
        attrs: Attributes * 
        innerValue: InnerValue
    | HtmlList of 
        listType: ListType * 
        headers: string list option * 
        itemCode: RenderingCode list
    | Sequence of RenderingCode list
    | Hole of FieldHole
  \end{lstlisting}
\end{listing}

\section{Combined traversal}

