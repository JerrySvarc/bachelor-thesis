\chapter*{Conclusion}
\addcontentsline{toc}{chapter}{Conclusion}


This thesis presents Data-driven UI, a novel prototype system that combines low-code and data-driven approaches to web application development. Our work explores a unique point in the design space of low-code programming systems, demonstrating what a minimal yet cleanly designed system satisfying our core principles can achieve.

The key contributions of this work include a hole-based approach for incremental UI creation, a method of structural referencing to combine UI elements with input data, and the implementation of core design principles: data-driven approach, low-code editing, and code generation capabilities. These features collectively enable a guided process of UI creation based on concrete data, which can potentially reduce the cognitive load on developers.

Our evaluation, conducted through benchmarks from the 7GUIs framework, demonstrates several strengths of our approach. The Counter task implementation showcased the system's ability to rapidly create functional UIs with minimal code, highlighting the potential of our data-driven, low-code approach. However, the challenges encountered with the Temperature Converter task revealed limitations in handling complex state changes and calculations, pointing to areas for future improvement.

The Data-driven UI system demonstrates notable benefits, including guided UI element creation, mitigation of vendor lock-in through code generation, and improved interoperability with existing systems. These advantages address some common pitfalls of low-code systems identified in previous research.

However, our approach also faces limitations. The necessity for upfront data preparation, the rigid element structure due to structural referencing, and potential performance issues with large datasets are areas that require further research and development.

Looking forward, several promising avenues for future work emerge. These include conducting empirical user studies, exploring solutions to the rigid element structure issue, improving the handling of complex data structures, expanding the UI component library, developing a low-code interface for implementing custom behaviors, and optimizing performance for large-scale applications.

In conclusion, while the Data-driven UI system is a prototype with limitations, it demonstrates the potential of combining low-code and data-driven approaches in web application development. As web applications continue to grow in complexity and importance, approaches like the one presented in this thesis may play a crucial role in simplifying and accelerating the development process. This work contributes to the evolving field of low-code development platforms and offers a new perspective on creating user interfaces based on concrete data.

