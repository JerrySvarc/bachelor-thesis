\chapter{Implementation}
\label{chap:implementation}

In this Chapter, we describe the practical realization of the design principles described in Chapter~\ref{chap:design} and the core logic described in Chapter~\ref{chap:corelogic}.
We will explore the technologies chosen, the system's architecture, and key implementation details of the Data-driven low-code programming system prototype.


\section{Technologies}
\label{sec:technologies}

The programming system is a browser-based client-side application providing a graphical user interface.
We decided to implement the system as a browser-based client-side application rather than a traditional desktop application.
We made this decision to allow greater cross-platform compatibility and the ability to see the preview of the created web applications in a browser-based environment.
We also wanted to implement the system using a programming language with functional programming capabilities and a strong ecosystem, which narrowed the number of possible technologies.
The system is implemented using the following key technologies:
\begin{itemize}
	\item \textbf{F\#:} The \citet{fsharp} programming language is used to implement the entire application, including the core logic and the user interface, chosen for its strong type system and functional programming capabilities.
	\item \textbf{Fable:} The \citet{fable} compiler, briefly described in Section~\ref{sub:Fable}, compiles the F\# source code to JavaScript, enabling browser-based execution and using technologies from the JavaScript ecosystem.
	\item \textbf{React:} The \citet{feliz} library provides a domain-specific language (DSL) for building \emph{React} user interface components in F\#.
	\item \textbf{Elmish:} \citet{elmish} is a library used to enable the creation of Elmish style applications in F\#, which follow the MVU pattern described in Section~\ref{sub:elmish}.
	\item \textbf{Tailwind:} We use the \citet{tailwind} CSS framework for the layout and styling of the UI components of the application, which provides composable CSS classes and enables high customizability of the UI.
	\item \textbf{SimpleJson:} The \citet{simpleJson} library is used to parse the input JSON data into the internal representation described in Section~\ref{sub:json}.
\end{itemize}

\subsection{Alternatives}

\section{System architecture}
\label{sec:appArch}
The implementation consists of different F\# \emph{modules}, each implementing different system functionality.
We can divide the modules into three categories based on the functionality they provide:
\begin{enumerate}
	\item \textbf{Core logic modules}
	\item \textbf{User interface modules}
	\item \textbf{Utility modules}
\end{enumerate}

Each module comprises different functions and or type definitions.
In our implementation, we divide the type definitions and functions into separate modules, similar to the separation of \emph{Domain} and \emph{Infrastructure} layers of the \emph{Domain-driven} architecture.
The modules containing the type definitions define the system's domain and have either none or a small number of external dependencies.
The modules containing functions then provide functionality over this domain.


\section{Core logic implementation}

\section{User interface}
\label{sec:ui}

\section{Challenges and Solutions}
\section{Summary}
