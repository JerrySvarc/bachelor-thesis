\chapter{Implementation}
\label{chap:implementation}

Transitioning from concepts to reality is a complex process involving many steps and decisions.
As we explore the implementation of our \emph{Data-driven UI} prototype programming system, we aim to provide a comprehensive view of how the concepts described in previous chapters materialize in practice.

We will describe the practical realization of the design principles described in Chapter~\ref{chap:design} and the core logic described in Chapter~\ref{chap:corelogic}.
We will explore the technologies chosen, the system's architecture, and key implementation details of \emph{Data-driven UI} programming system.



\section{Technologies}
\label{sec:technologies}
To provide context for the implementation, we'll first examine the technologies used in developing the \emph{Data-driven UI} system, as our technological choices significantly shaped the design and functionality of the resulting application.

We decided to create the \emph{Data-driven UI} editor in a from of a browser-based client-side application providing a graphical user interface, rather than a traditional desktop application.
We made this decision to allow greater cross-platform compatibility and the ability to see the preview of the created web applications in a browser-based environment.
We also wanted to implement the system using a programming language with functional programming capabilities and a strong ecosystem, which narrowed our range of suitable options.
The system is implemented using the following key technologies:
\begin{itemize}
	\item \textbf{F\#:} The \citet{fsharp} programming language is used to implement the entire application, including the core logic and the user interface, chosen for its strong type system and functional programming capabilities.
	\item \textbf{Fable:} The \citet{fable} compiler, briefly described in Section~\ref{sub:Fable}, compiles the F\# source code to JavaScript, enabling browser-based execution and using technologies from the JavaScript ecosystem.
	\item \textbf{React:} The \citet{feliz} library provides a domain-specific language (DSL) for building \emph{React} user interface components and applications in F\#.
	\item \textbf{Elmish:} \citet{elmish} is a library used to enable the creation of Elmish style applications in F\#, which follow the MVU pattern described in Section~\ref{sub:elmish}.
	\item \textbf{Tailwind:} We use the \citet{tailwind} CSS framework for the layout and styling of the UI components of the application, which provides composable CSS classes and enables high customizability of the UI elements.
	\item \textbf{SimpleJson:} The \citet{simpleJson} library is used to parse the input JSON data into the internal representation described in Section~\ref{sub:json}.
	\item \textbf{SAFE stack template:} We use the \citet{safestack} template's \emph{Build} project, which provides scripts for building web applications built in F\#.
\end{itemize}


\subsection{Alternatives}
While we selected the technologies mentioned in the previous Section for our implementation, for several of them we considered using alternative technologies:
\begin{itemize}
	\item \textbf{Programming Language:} Instead of F\#, we could have used other functional programming languages such as Haskell or OCaml.
	      These languages also offer strong type systems and functional programming capabilities.
	      However, F\# was chosen for its seamless integration with the .NET ecosystem, high-quality development tools and documentation, and its ability to be compiled to JavaScript via Fable.

	\item \textbf{CSS Framework:} We considered using alternative CSS frameworks, such as Bulma or Bootstrap, which provide pre-made styled components.
	      We chose to use Tailwind instead, as the composable styling classes allow for a more direct approach to styling UI elements, instead of trying to adapt and modify the pre-made components.

	\item \textbf{JSON Parsing:} Instead of Fable.SimpleJson, we could have used the closest alternative library called \citet{thoth}.
	      The main strenght of this library is the ability to create custom JSON \emph{encoders} and \emph{decoders}.
	      However, as we need the ability to parse JSON data of abitrary strucuture, SimpleJson's lightweight nature and its internal representation of the parsed data made it our preffered choice.
\end{itemize}
While these alternatives have their strengths, our chosen technologies provided the best balance of functional programming capabilities, browser compatibility, and ecosystem support for our specific requirements.



\section{System architecture}
\label{sec:appArch}
Before we describe the implementation specifics, we must first describe the \emph{architecture} of the implementation.
The implementation is divided between different F\# \emph{modules}, each providing different functionality.
We define two main modules named \emph{Core Logic} and \emph{Editor}, which contain sub-modules implementing the specific functionality.
These modules can be described as follows:
\begin{enumerate}
	\item \textbf{Core Logic module:} Contains modules responsible for implementing the Core logic described in Chapter~\ref{chap:corelogic}.
	\item \textbf{Editor module:} Modules focused on implementing the user interface of the programming system and its functionality following the Elmish architecture described in Section~\ref{sub:elmish}.
	      They use and depend on the functionality provided by the CoreLogic module.
\end{enumerate}
We will describe the main modules in greater detail in the following Sections.

Each sub-module comprises various functions or type definitions.
In our implementation, we separate type definitions and functions into distinct modules, similarly to the separation of \emph{Domain} and \emph{Infrastructure} layers of the \emph{Domain-driven} architecture.
The modules containing the type definitions define the system's domain and have either none or a small number of external dependencies.
The modules containing the functions implement the behaviors and operations that manipulate and utilize the domain types.

This approach inherently makes the application more easily extensible.
For example, we can add new functionality to the Editor module without the need to change the implementation of the Core Logic sub-modules.

\subsection{Core Logic Module}
The Core Logic module is the first main module of the \emph{Data-driven UI's} implementation.
It forms the foundation of our \emph{Data-driven UI} system, implementing the types and operations described in detail in Chapter~\ref{chap:corelogic}.
Figure~\ref{fig:core-logic-structure} illustrates the overall structure of this module.
We divide the implementation into the following sub-modules:
\begin{itemize}
	\item \textbf{Types:} This module comprises sub-modules that define the fundamental data structures of our system:
	      \begin{itemize}
		      \item \textbf{RenderingTypes:} Contains definitions for types such as \emph{RenderingCode}, \emph{CustomWrapper}, and other related types.

	      \end{itemize}

	\item \textbf{Operations:} This module includes sub-modules that implement various operations on the types defined in the Types module:
	      \begin{itemize}
		      \item \textbf{RenderingCode:} Implements crucial operations such as the \emph{replace} function, which modifies the RenderingCode AST.
		            This module is central to manipulating our UI representation.

		      \item \textbf{DataRecognition:}  Handles the mapping process between input data and our internal UI element representation.
		            It includes the \emph{recognizeJson} function, detailed in Section~\ref{sec:mapping}, which is key to our data-driven approach.

		      \item\textbf{CodeGeneration:} Provides functionality to generate textual representations of created web applications from RenderingCode ASTs.
		            This module bridges our internal representations with output code.
	      \end{itemize}
\end{itemize}


\begin{figure}[htbp]
	\centering
	\begin{tikzpicture}[
			node distance = 0.5cm and 0.5cm,
			mainblock/.style = {rectangle, draw, fill=gray!20,
					text width=6em, text centered, rounded corners, minimum height=3em},
			typeblock/.style = {rectangle, draw, fill=blue!20,
					text width=6em, text centered, rounded corners, minimum height=3em},
			opblock/.style = {rectangle, draw, fill=green!20,
					text width=6em, text centered, rounded corners, minimum height=3em},
			line/.style = {draw, -stealth},
		]
		% Main Core Logic node
		\node [mainblock, text width=10em] (core) {Core Logic};

		% Types
		\node [typeblock, below left=of core] (types) {Types};
		\node [opblock, below=of types] (renderingtypes) {Rendering\\Types};

		% Operations
		\node [typeblock, below right=of core] (operations) {Operations};
		\node [opblock, below left=of operations] (rendering) {Rendering\\Code};
		\node [opblock, below=of operations] (recognition) {Data\\Recognition};
		\node [opblock, below right=of operations] (generation) {Code\\Generation};

		% Draw arrows
		\path [line] (core) -- (types);
		\path [line] (core) -- (operations);
		\path [line] (types) -- (renderingtypes);
		\path [line] (operations) -- (rendering);
		\path [line] (operations) -- (recognition);
		\path [line] (operations) -- (generation);


	\end{tikzpicture}
	\caption{Core Logic Module Structure}
	\label{fig:core-logic-structure}
\end{figure}

This modular structure allows for clear separation of concerns, with types and operations distinctly categorized.
It improves the extensibility and maintainablity of the Core Logic functionality.

\subsection{Editor module}
The Editor module is the second main module of the \emph{Data-driven UI's} implementation.
It is responsible for implementing the programming system's UI and functionality.
It facilitates tools from the Core Logic module and technologies described in Section~\ref{sec:technologies} such as Elmish, to provide an interactive low-code editor environment.
Figure~\ref{fig:editor-module-structure} illustrates the structure of the module.

We divide the module into the following sub-modules:
\begin{itemize}
	\item \textbf{Types:} This module defines the core data structures that represent the state and operations of our Elmish-style application, as detailed in Section~\ref{sub:elmish}.
	      \begin{itemize}
		      \item \textbf{EditorDomain:} Encapsulates the domain types that represent the internal state of the Data-driven UI application.
		      \item \textbf{PageEditorDomain:} Defines specialized domain types focused on representing the internal state of the Page Editor component.
	      \end{itemize}

	\item \textbf{Utilities:} Provides a collection of essential helper functions that support various aspects of the application:
	      \begin{itemize}
		      \item Icon importing and management.
		      \item File upload handling and processing.
		      \item JSON data parsing and serialization, leveraging the \citet{simpleJson} library.
		      \item Importing the libraries necessary to use the \citet{codemirror} editor component.
	      \end{itemize}


	\item \textbf{Components:} Implements a suite of custom React components that form the interactive user interface:
	      \begin{itemize}
		      \item \textbf{CustomRendering:} Implements dynamically rendering previews of the RenderingCode AST, coupled with interactive modification menus that adapt to the current editing context.
		      \item \textbf{OptionComponents:} Provides components serving as context menus to allow editing functionality for the individual RenderingCode elements.
		      \item \textbf{EditorComponents:} Provides a set of versatile components that deliver general editor functionality, including a tab management system inspired by industry-standard editors like Visual Studio Code.
		      \item \textbf{PageEditorComponents:} Offers specialized components designed for creation, real-time preview, and efficient modification of the RenderingCode AST.
	      \end{itemize}
\end{itemize}


\begin{figure}[htbp]
	\centering
	\begin{tikzpicture}[
			node distance = 1cm,
			mainblock/.style = {rectangle, draw, fill=gray!20,
					text width=8em, text centered, rounded corners, minimum height=3em},
			subblock/.style = {rectangle, draw, fill=blue!20,
					text width=7em, text centered, rounded corners, minimum height=2.5em},
			subsubblock/.style = {rectangle, draw, fill=green!20,
					text width=6em, text centered, rounded corners, minimum height=2em},
			line/.style = {draw, -stealth},
		]
		% Main Editor Module node
		\node [mainblock] (editor) {Editor Module};

		% Main sub-modules

		\node [subblock, below=of editor] (utilities) {Utilities};
		\node [subblock, left=of utilities] (types) {Types};
		\node [subblock, right=of utilities] (components) {Components};

		% Draw arrows
		\foreach \i in {types,utilities,components}
		\path [line] (editor) -- (\i);

	\end{tikzpicture}
	\caption{Editor Module Structure}
	\label{fig:editor-module-structure}
\end{figure}


\section{Main Features}
\label{sec:features}
Now that we have defined the implementation's modular architecture and structure, we can describe the specifics of the implementation.
As the implementation comprises many different type definitions and functions of various complexity,
we will briefly describe the system's \emph{main features} that implement the design principles described in Chapter~\ref{chap:design}.

\subsection{User interface}

The integral component of our programming system is the low-code user interface, as it realizes one of the design principles described in Chapter~\ref{chap:design}.
It provides tools for creation and modification of UI elements through the use of context menus.
It also shows the preview of the already created UI elements.

To implement the programming system, we use tools and technologies previously described in Section~\ref{sec:technologies}.
One of these tools is the \citet{feliz} library, used for creating composable React UI components in F\#.
Another important tool is the \citet{elmish} library, which provides abstractions that enable the creation of \emph{Elmish-style} applications in F\#.
We can also create React components that use the Elmish-style architecture as part of their internal representation of state and operations
by using the \emph{React.useElmish} hook provided by the \emph{Feliz.UseElmish} library.

\subsubsection{Application-Level Elm Architecture}

At the top level, our application follows the Elm architecture described in Section~\ref{sub:elmish}.
The primary application state is represented by the \emph{Model} type, and we define corresponding \emph{View} and \emph{Update} functions to manage this state.
We can see the \emph{Model} type representing the entire application's state in Program~\ref{fig:appmodel}.

\begin{listing}[htbp]
	\caption{The Model type representing the state of the \emph{Data-driven UI} system.}
	\label{fig:appmodel}
	\begin{lstlisting}
type Model = {
    Pages: Map<Guid, Page>
    OpenTabs: Tab list
    ActiveTabId: Guid option
    IsSidebarOpen: bool
}
    \end{lstlisting}
\end{listing}

The representation shows that we provide tabbing functionality similar to existing programming systems, such as \emph{Visual studio code}.
This means the user can create and edit multiple Pages simultaneously, switching between them as needed.

\subsubsection{Component-Level Elmish Architecture}

For specific components that require more complex, localized state management,
we utilize the previously described Feliz.UseElmish library.
This approach allows us to create Elmish-style React components
that have their own local state, update logic, and view functions while still being part of the greater Elm-style application.

For example, the \emph{PageEditor} components use the \emph{Feliz.UseElmish} library to manage their internal state.
This structure allows the PageEditor to maintain its own state and update logic,
making it more modular and accessible to reason about while still communicating with the main application when necessary through the use of Elmish \emph{Commands}.
We can see the \emph{PageEditorModel} representing the state of the PageEditor React component in Program~\ref{fig:editorModel}.

\begin{listing}[htbp]
	\caption{The PageEditorModel type representing the state a PageEditor component.}
	\label{fig:editorModel}
	\begin{lstlisting}
type PageEditorModel = {
    PageData: Page
    FileUploadError: bool
    IsPreview: bool
}
    \end{lstlisting}
\end{listing}

\subsubsection{User Interface layout}

\xxx{//TODO: Screenshot of the user interface}

The user interface is composed of multiple React components some managed by the top-level Elmish application, while others employing the
component-level architecture.

We can see the layout of the \emph{Data-driven UI's} interface in \xxx{Figure~\ref{fig:}}
Located on the left is a collapsable side-menu used to open created pages for editing.
On the top we can see a panel displaying tabs for currently open pages.
The state and operations of these components is managed by the main Elmish application.

Under the tab panel we can see the main editor window.
The left side of the window is used to provide the modification and preview functionality explained in more detail in the following Section.
On the right side we see a window for creating and modifying custom JavaScript handler functions.
The state and operations of these components is managed by the individual PageEditor components.

\subsection{Custom Rendering}

To implement the Low-code approach described in Chapter~\ref{chap:design} combined with the Data-driven approach described in Chapter~\ref{chap:corelogic},
we define a \emph{Higher-order} recursive function named \emph{renderingCodeToReactElement}.
This function performs the simultaneous traversal of the RenderingCode and JSON ASTs described in Chapter~\ref{chap:corelogic}.

The arguments of the \emph{renderingCodeToReactElement} function are the following:
\begin{itemize}
	\item \emph{code}: The RenderingCode to be rendered
	\item \emph{path}: The current path in the RenderingCode AST
	\item \emph{json}: The corresponding JSON data
	\item \emph{name}: A string identifier for the current element corresponding to a JSON field of the same path
	\item \emph{dispatch}: A function to dispatch messages to the Elmish update function
\end{itemize}

The function recursively traverses the RenderingCode AST, matching each type of RenderingCode to its appropriate rendering function.
We then define a rendering function for each type of RenderingCode.
Each rendering function combines the RenderingCode structure with the corresponding JSON data to create a preview of the element.
The function also interweaves option menus with the rendered elements when, allowing for editing of the rendered structure.
Error handling is implemented throughout the rendering process to handle mismatches between the RenderingCode and JSON structures.

This function provides the main functionality of our implementation and effectively combines the functionality from the Core logic module,
with the low-code approach.

\subsection{Code generation}

One of our core design principles, as outlined in Chapter~\ref{chap:design},
is the ability to generate a textual representation of the created application.

To implement this functionality, we define a recursive function named \emph{generateHtml}.
This function operates in a manner similar to the \emph{renderingCodeToReactElement} function described in the previous section,
performing a simultaneous traversal of the RenderingCode and JSON ASTs as detailed in Chapter~\ref{chap:corelogic}.

For each type of RenderingCode, the function generates appropriate HTML:
\begin{itemize}
	\item \texttt{HtmlElement} is translated into its corresponding HTML element
	\item \texttt{HtmlList} generates HTML lists based on the list type
	\item \texttt{HtmlObject} creates a structured representation of the object's properties and based on the object type
	\item \texttt{Hole} elements are represented as comments in the generated HTML, maintaining the structure for potential future updates
	\item \texttt{CustomWrapper} and \texttt{CustomElement} types are handled to generate the user-defined elements
\end{itemize}

A key feature of this function is its ability to automatically populate the generated HTML elements with the corresponding JSON data,
ensuring that the textual representation accurately reflects both the structure and content of the created application.

The generated HTML is formatted with proper indentation for readability,
making it easier for developers to understand and potentially modify the code outside of the low-code environment.
This approach not only facilitates the transition from low-code to traditional development when necessary but also promotes transparency in the development process.

\section{Building and Deployment}
\label{sec:build}
Now that we have described the implementation of the \emph{Data-driven UI} system, we must describe how the application is built and deployed.
As we mentioned in Section~\ref{sec:technologies}, we use the \emph{Build} project provided by the \citet{safestack} template.
The original SAFE stack \emph{Build} project supports building \emph{full-stack} applications written in F\#, and we modified it to remove the support for building
the \emph{Shared} and \emph{Server} projects, as our programming system only consists of a client-side application.

\subsection{Build Tools and Libraries}
The SAFE stack \emph{Build} project eploys different specialized tools to make building F\# application easier.
The two main tools used by the \emph{Build} project are the following:
\begin{itemize}
	\item \textbf{FAKE (F\# Make):} A build automation tool for projects written in F\#, used to define build targets and their individual build steps.
	\item \textbf{Vite:} A web development tool for building JavaScript client-side applications, which provides a development server environment with Hot Module Replacement capability.
\end{itemize}


The pre-requisites are needed to build the application:
\begin{itemize}
	\item \textbf{.NET Core SDK}
	\item \textbf{Node 20}
\end{itemize}

\subsection{Build targets}
The \emph{Build} project provides multiple different build targets.
Each provides a different build result and comprises a different sequence of build steps.

The \emph{Build} project defines the follwing targets:
\begin{itemize}
	\item \textbf{Run:} Configuration for running a local development server with HMR capabilities.
	      Uses the Fable compiler to compile the source code into JavaScript and uses Vite to run the development server.
	\item \textbf{Bundle:} Target used for bundling of the application for deployment.
	      Uses the Fable compiler to compile the source code into JavaScript and uses Vite to
	      bundle and optimize the application assets, including the JavaScript program, CSS, and static files.
	\item \textbf{RunTests:} Configuration for running the unit tests.
	\item \textbf{Format:} Uses the \emph{Fantomas} F\# code formatter, to format the source code.
\end{itemize}
In order to build a specific target, we use the following command:
\begin{lstlisting}[language=bash]
  $ dotnet run [Target] 
\end{lstlisting}
It consists of the \emph{dotnet run} command, followed by the name of the target we wish to build.
It is important to mention, that the target names are case-sensitive.


\section{Testing}
\label{sec:testing}
To decide if the implemented operations perform as expected, we decided to create a separate project named \emph{Tests}.
We implemented \emph{Unit tests} for various operations from the Core Logic module, such as the \emph{replace} function from the \emph{CoreLogic.Operations.RenderingCode} module, and the code generation capability from the \emph{CoreLogic.Operations.CodeGeneration} module.

We use the testing library named \citet{mocha}, which provides useful tools to implement the testing functionality.
It provides a browser-based testing environment with a graphical user interface displaying the individual test cases and their results.
It also allows composing related \emph{test cases} into \emph{test lists}, improving the readability and maintainablity of the tests.

The structure of the individual test cases follows the \emph{Arange Act Assert} (AAA) unit test pattern.
It involves preparing the required data, then performing some operation using this data, and then asserting that the operation has the expected result.
We can see an example test case in Program~\ref{fig:test}.
We can see the name of the test case and its implementation which follows the \emph{AAA} pattern.

\begin{listing}[htbp]
	\caption{An example test case testing the functionality of the \emph{replace} function.}
	\label{fig:test}
	\begin{lstlisting}
testCase "Replace Hole"
  <| fun _ ->
      let original = Hole(Named "placeholder")
      let replacement = HtmlElement(Div, [], Empty, [])
      let result = replace [] replacement original
      Expect.equal result replacement "Should replace Hole at root level"
  \end{lstlisting}
\end{listing}


\section{Summary}

In this chapter, we described the practical implementation of our Data-driven UI programming system,
translating the theoretical concepts and core logic into a functional prototype.
The key points covered in this chapter are:

\begin{itemize}
	\item In Section~\ref{sec:technologies}, we outlined the technologies used in our implementation, including F\#, Fable, React, Elmish, and others.
	      We also discussed alternative technologies considered and our rationale for our choices.

	\item In Section~\ref{sec:appArch}, we presented the system architecture,
	      detailing the modular structure of our implementation.
	      We described the Core Logic and Editor modules, explaining how they interact and their respective responsibilities.

	\item In Section~\ref{sec:features}, we explored the main features of our system, focusing on:
	      \begin{itemize}
		      \item The user interface, which combines an application-level Elm architecture with component-level Elmish architecture for more complex elements.
		      \item The custom rendering system, which implements the low-code and data-driven approaches through the \texttt{renderingCodeToReactElement} function.
		      \item The code generation capability, which translates our internal representations into HTML and JavaScript.
	      \end{itemize}

	\item In Section~\ref{sec:build}, we described the build and deployment process, detailing the tools used and the available build targets.

	\item In Section~\ref{sec:testing}, we outlined our testing approach, explaining how we use unit tests to verify the correctness of our core operations.
\end{itemize}






