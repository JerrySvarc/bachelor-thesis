\chapter*{Introduction}
\addcontentsline{toc}{chapter}{Introduction}
\xxx{
	Since the creation of the World Wide Web, developers have created web applications mainly by writing and modifying text-based code.
	This approach is mainly used even today, although made easier by many different tools, languages, frameworks, and programming systems.
	//TODO
}

\section*{Problem description}
Low-code programming sytems are systems that aim to significantly reduce the amount of code needed to create certain applications \cite{Pinho_Aguiar_Amaral_2023}.
They can also help less experienced developers, by providing them with an easy-to-use interface, through which the developers can create program artifacts.

This thesis presents a data-driven approach for creating user interface elements of web applications.
We aim to create a low-code programming system based on a data-driven approach.
The approach involves first providing data to the application, which serves as the model of the application.
Then, the programming system aids the developer in creating the user interface by providing suggestions based on the structure and type of the data.
After creating the user interface, the developer can download the text-based representation of the application and modifying the behavior of these components and how they change the data itself.
This system will allow the developers to create the user interface elements of the application without the need to write a large amount of boilerplate code.

\section*{Goals}
The main goals of this thesis are the following:
\begin{enumerate}
	\item Explore the applicability of the low-code programming system employing the data-driven approach to the creation of UI elements of web applications:
	      \begin{itemize}
		      \item Define the \emph{Data-driven} approach.
		      \item Define the approach in context of creating web applications based on concrete data.
		      \item Describe the design principles behind the approach.
		      \item Present a description of combining the created UI elements with the input data.
		      \item Evaluate the feasability of this approach in the context of the prototype application.
	      \end{itemize}
	\item Create a working \textbf{prototype programming system} implementing the data-driven approach:
	      \begin{itemize}
		      \item Create the prototype application that should:
		            \begin{itemize}
			            \item Be browser-based.
			            \item Provide a low-code interface.
			            \item Allow the user to upload JSON data.
			            \item Provide context menus based on the provided data and aid the user in creating the UI elements.
			            \item Show a preview of the user-created parts of the web application.
			            \item Allow the user to download the created web application in a text-based form.
		            \end{itemize}

		      \item Benchmark the prototype application on the following task:
		            \begin{itemize}
			            \item Counter task from the 7GUIs Benchmark
			            \item Temperature converter task from the 7GUIs Benchmark
			            \item CRUD task from the 7GUIs Benchmark
			            \item Simple To-do list application
		            \end{itemize}

	      \end{itemize}
\end{enumerate}

\section* {Contributions and Outline}
\begin{itemize}
	\item In Chapter \ref{chap:design}, we describe the design principles which influenced the design of the proposed programming system.
	\item In Chapter \ref{chap:corelogic}, \xxx{//TODO}
	\item In Chapter \ref{chap:implementation}, we describe the implementation specifics of the prototype application.
	\item In Chapter \ref{chap:walktrough},we showcase a creation of a user interface for a simple to-do list application.
	\item In Chapter \ref{chap:discussion}, \xxx{//TODO}
\end{itemize}
