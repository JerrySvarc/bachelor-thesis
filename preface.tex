\chapter*{Introduction}
\addcontentsline{toc}{chapter}{Introduction}
\xxx{
	Since the creation of the World Wide Web, developers have created web applications mainly by writing and modifying text-based code.
	This approach is mainly used even today, although made easier by many different tools, languages, frameworks, and programming systems.
	//TODO
}

\section*{Problem description}
Low-code programming sytems are systems that aim to significantly reduce the amount of code needed to create certain applications \cite{Pinho_Aguiar_Amaral_2023}.
They can also help less experienced developers, by providing them with an easy-to-use interface, through which the developers can create program artifacts.

This thesis presents a novel data-driven approach for creating the client-side part of dynamic single-page applications.
We aim to create a low-code programming system based on a data-driven approach.
The data-driven approach involves first providing data to the application, which serves as the model of the application.
Then, the programming system aids the developer in creating the user interface by providing suggestions based on the structure and type of the data.
After creating the user interface, the developer can download the text-based representation of the application and modifying the behavior of these components and how they change the data itself.
This system will allow the developers to create the \emph{User Interface} of the application without the need to write a large amount of boilerplate code.

\section*{Goals of the thesis}
//TODO

\section* {Contributions and Outline}
\begin{itemize}
	\item In Chapter \ref{chap:walktrough},we showcase a creation of a user interface for a simple to-do list application.
	\item In Chapter \ref{chap:design}, we describe the design principles which influenced the design of the proposed programming system.
	\item In Chapter \ref{chap:core-logic}, we describe the type system, describe the data analysis that the system makes, and describe how the system interprets the data provided to it.
	\item In Chapter \ref{chap:implementation}, we describe the implementation specifics of the prototype application.
	\item In Chapter \ref{chap:discussion}, //TODO
\end{itemize}
