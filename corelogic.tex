\chapter{Core Logic}
\label{chap:corelogic}


This chapter describes the core logic of the programming system.
Firstly, we describe the hole-based approach applied to the creation of elements.
After that, we describe the type system, mainly the internal representation of the UI elements and necessary operations on these types.
We describe mapping the input data types to specific types representing the UI elements.
Lastly, we describe how the system combines the input data and the corresponding UI elements.
All example Programs shown in this Chapter are written in the F\# programming language.

\section{Hole-based Approach}
\label{sec:hole-based}
The primary motivation of this approach is to allow \emph{incremental creation} of UI elements.
The elements are created based on the provided data and its structure.
Each element corresponds to a data value.
We need the ability to represent data values that have yet to be used to create UI elements.
To address this need, we define a type named \emph{Hole} as follows:
\begin{defn}[Hole type definition]
	A Hole is a placeholder type representing a UI element that is yet to be created despite its existing corresponding data. It serves as a temporary stand-in during the incremental construction of a user interface.
\end{defn}

We then define the \emph{incremental creation process} as a \emph{sequence of discrete opreations}.
Each operation is either a \emph{modification} of an existing UI element or a \emph{replacement} of a Hole element with a new UI element based on the corresponding data.
This discrete approach allows the system to perform different tasks after each operation.
These tasks include updating the UI element preview, analyzing the input data, or providing new modification menus and options.
The system can also analyze the existing UI elements and perform other operations at this time.


\section{Type System and Mapping}
\label{sec:types}
In this section, we describe the types which the system uses to represent the input data and types used to represent the created UI elements.
We also define the mapping between the input data and the UI element representation.


\subsection{JSON Type}
\label{sub:json}
The creation process starts by providing \emph{JSON} data to the system.
In order to use this data, we parse it and create an internal representation.
We define a discriminated union type named \emph{JSON}.
This type is the foundation for the Abstract Syntax Tree representation of the JSON data.
We can see the type definition in Program \ref{fig:json}.


Each node represents a JSON value in the input data.
The types of nodes can be divided into two categories:
\begin{itemize}
	\item {\textbf{Collections:} The first category contains types representing a \emph{collection} of other values. This category includes the types \emph{JObject} and \emph{JArray}.
	      We define the two collection types as follows:
	      \begin{itemize}
		      \item \textbf{JObject:} It is based on the JSON Object type and represents a collection of different JSON types. The original ordering of the inner elements is \emph{ignored}.
		      \item \textbf{JArray:} It is based on the JSON List type and represents a collection of JSON values of the \emph{same type and structure}. The original ordering of the inner elements is \emph{preserved}.
	      \end{itemize}
	      }
	\item \textbf{Primitives:} The second category contains types representing data primitives such as numerical values, boolean values,
	      a string of text, or the null value.
\end{itemize}

\begin{listing}[htbp]
	\caption {JSON type}
	\label{fig:json}
	\begin{lstlisting}
type Json =
  | JNumber 
  | JString 
  | JBool
  | JNull
  | JArray 
  | JObject 
  \end{lstlisting}
\end{listing}

\subsection{RenderingCode Type}
To facilitate the creation of UI elements based on the JSON type, we define a type named \emph{RenderingCode}.
The \emph{RenderingCode} is a discriminated union type used to represent the UI elements.
Similarly to the JSON type, each case represents a type of an HTML element with a corresponding mapping to the JSON type.
This type is the foundation for the \emph{Abstract Syntax Tree} representation of UI elements.
We can see the type definition in Program~\ref{fig:renderingCode}.
\begin{listing}[htbp]
	\caption {RenderingCode type}
	\label{fig:renderingCode}
	\begin{lstlisting}
type RenderingCode =
  | HtmlElement 
  | HtmlList 
  | HtmlObject 
  | Hole 
  \end{lstlisting}
\end{listing}

We define each case type and what it consists of as follows:
\begin{itemize}
	\item \textbf{HtmlElement:} Represents a single HTML element. It consists of:
	      \begin{itemize}
		      \item \emph{Tag}: Denotes the HTML tag (e.g., div, p, span). \item \emph{InnerValue}: Represents the content of the element defined as seen in Program~\ref{fig:inner}.
		            Where \emph{Data} indicates dynamic content from input JSON, \emph{Empty} represents an empty element, and \emph{Constant} holds static, developer-defined content.
		      \item {\emph{Attributes}: A collection of key-value pairs representing HTML attributes.
		            It consists of a string key and an InnerValue value. This representation allows for the selection of an attribute's value from the input data.
		            }

	      \end{itemize}
	      \begin{listing}[htbp]
		      \caption{InnerValue type definition}
		      \label{fig:inner}
		      \begin{lstlisting}
type InnerValue =
  | Data 
  | Empty
  | Constant
                  \end{lstlisting}
	      \end{listing}

	\item \textbf{HtmlList:} Represents a collection of HTML elements of the same type corresponding to JSON arrays. It includes:
	      \begin{itemize}
		      \item \emph{ListType}: Specifies the list type (e.g., unordered, ordered, table).
		      \item \emph{itemCodes}: A list of RenderingCode elements representing list items.

	      \end{itemize}

	\item \textbf{HtmlObject:} Represents a collection of diverse HTML elements derived from JSON object. It consists of:
	      \begin{itemize}
		      \item \emph{ObjectType}: Defines the type of the object (e.g., div, span, Section).
		      \item \emph{Elements}: A list of RenderingCode elements representing the object's contents.
	      \end{itemize}

	\item {\textbf{Hole:} Represents a placeholder in the UI element structure as defined in Section~\ref{sec:hole-based}, allowing for dynamic element creation.
	      }
\end{itemize}


Each RenderingCode type, with the exception of the type \emph{Hole}, also provides the ability to add custom event handling.
Each elements includes a list of user-defined event handlers.
Each event handler consists of the name of the event, and a script that is executed.
This script is user-defined and implemented in JavaScript.

\subsection{Data Mapping: JSON to RenderingCode}
\label{sec:mapping}
Transforming JSON data into UI elements requires systematic mapping between JSON and RenderingCode types.
This mapping forms the core of our UI generation system, allowing us to convert arbitrary JSON input into a structured representation of UI elements.
Each JSON type has a corresponding representation in RenderingCode:
\begin{itemize}
	\item JObject maps to HtmlObject
	\item JArray maps to HtmlList
	\item JNull, JString, JNumber, and JBool map to HtmlElement
\end{itemize}

The \emph{incremental creation process} described in Section~\ref{sec:hole-based} involves dynamically creating replaceable \emph{Hole} elements.
We perform this creation lazyly and incorporate it into the mapping process.
A \emph{Hole} element is created for each inner element of the collection types described in Section~\ref{sub:json}.


We call this mapping process the \emph{recognition} of the JSON value and define the mapping function named \emph{recognizeJson} in Program~\ref{fig:recognizeJson}.

\begin{listing}[htbp]
	\caption {JSON to RenderingCode mapping}
	\label{fig:recognizeJson}
	\begin{lstlisting}
let recognizeJson (json) =
    match json with
    | JArray array -> 
        //create a hole for each element of the array
        HtmlList containing the holes as elements
    | JObject obj ->
        //create a Hole for each element of the object 
        HtmlObject containing the holes as elements
    | JNull -> Hole(UnNamed)
    | _ -> HtmlElement 
  \end{lstlisting}
\end{listing}

\section{Incremental Creation Process}
\label{sec:creation}
We described the \emph{incremental creation process} briefly in Section~\ref{sec:hole-based} as a sequence of discrete operations.
In this Section, we extend this description and define the previously described operations in greater detail.
We divide the operations into two categories:
\begin{itemize}
	\item \textbf{User operations:} Operations performed by the user such as element creation, element modification, and providing data to the system.
	      User operations are performed through the provided GUI.
	\item \textbf{System operations:} Operations performed by the system in reaction to the User operations.
	      These operations include creating new RenderingCode elements, analyzing newly visited JSON values, modifying AST, rendering menus, and previewing elements.
\end{itemize}
This categorization allows us to divide the functionality and responsibilities of the system between different parts of the application.
For example, the GUI portion of the application is mainly responsible for accepting User operations, whereas other modules can implement different System operations.


\subsection{Traversal}
\label{sec:traversal}
An essential aspect of the creation process is how the system processes the JSON AST and helps the user build the RenderingCode AST.
The process involves traversing both the JSON and RenderingCode ASTs simultaneously.
We can do this because we create the RenderingCode AST based on the structure of the JSON AST.
This allows the system to dynamically create Hole elements for direct descendants of existing RenderingCode elements for which corresponding data exists, but the elements have not yet been created.

Another way to look at this process is that the RenderingCode AST is guiding our traversal, while the JSON AST inspires the structure of the RenderingCode AST.
The traversal begins from the roots of the ASTs, and we traverse the trees recursively.
Based on the type of the visited RenderingCode node and the mapping described in Section~\ref{sec:mapping}, we can perform different operations:
\begin{itemize}
	\item \textbf{HtmlElement:} The node is of type HtmlElement, which maps to a primitive JSON value. This means the JSON value has no descendants, and we can end the traversal.
	      The system then performs the operations of displaying a preview of the HtmlElement and a modification menu.
	\item \textbf{Hole:} The node is of type Hole, meaning the JSON value was visited before, but the user has not created the corresponding RenderingCode.
	      The presence of a Hole element also means that we cannot visit the potential descendants, and we can stop the traversal.
	      The system displays a menu to add a new element corresponding to the JSON value.

	\item \textbf{HtmlList and HtmlObject:} When the node represents a \emph{Collection} type of type HtmlList or HtmlObject, we recursively traverse all its descendants.
	      The system displays a modification menu for the element and continues the traversal.
\end{itemize}

After every User operation, the system performs this traversal to update the element preview and modification menus to reflect changes made to the RenderingCode AST.


\subsection{Hole Replacement and Modification}

User operations require finding a specific node inside the RenderingCode AST during the creation process.
We define a dynamically generated \emph{Path} for all nodes during the traversal process to find this specific node.
The \emph{Path} is a sequence of indices unique to every node in the AST.

\begin{listing}[htbp]
	\caption {Example: RenderingCode AST with corresponding paths}
	\label{fig:paths}
	\begin{lstlisting}
  HtmlObject [ //Path: []
      HtmlElement(Div, [], Empty) //Path: [0]
      HtmlList(UnorderedList, [ //Path: [1]
          HtmlElement(Li, [], Constant "Item 1") //Path: [1,0]
          HtmlElement(Li, [], Constant "Item 2") //Path: [1,1]
				])
		]
  \end{lstlisting}
\end{listing}

We can see the different paths for each element in Program~\ref{fig:paths}, which shows an example of RenderingCode AST.
The example AST consists of a root node of a type HtmlObject and elements contained within it.
We see that the root node has an empty path.
Then, we append the position of each element inside the collection to the Path.
This allows us to traverse the AST based on the specified Path and find the correct element using a recursive search function.

Using the dynamically created paths, we can replace a RenderingCode by following its specified Path through the RenderingCode AST.
We define a recursive function named \emph{replace} which navigates the AST using the provided path to locate and replace a specific element with a new one.
We can see a description of the potential implementation in Program~\ref{fig:replace}.

\begin{listing}[htbp]
	\caption {A function used to replace a RenderingCode inside the RenderingCode AST}
	\label{fig:replace}
	\begin{lstlisting}
let rec replace 
  (path: int list) 
  (replacementCode: RenderingCode) 
  (currentCode: RenderingCode) =
  match path with
  | [] -> replacementElement
  | head :: tail ->
      match currentCode with
      | HtmlList ->
          // recursively call the function on a collection element with the index equal to head 
          // return new HtmlList
      | HtmlObject ->
          // recursively call the function on an element with 
          // the key located inside the keys collection, that has index equal to head
          // return new HtmlObject
      | _ -> currentCode // remaining path, but the current element is not a collection
  \end{lstlisting}
\end{listing}

The User operations such as replacing a Hole with a new RenderingCode or modification of an existing RenderingCode correspond to a specific usage of the replace function.

\subsection{Creation of a single RenderingCode}

Now that all necessary operations are defined, we focus on the creation process of a single RenderingCode element based on the input data.
Creating a single RenderingCode element consists of multiple System operations and a single User operation.
We can see the sequence of operations in Figure~\ref{fig:element-creation}.

The process starts when the system visits a previously unvisited node of the JSON AST during the traversal described in Section~\ref{sec:traversal}.
The system automatically creates a corresponding Hole element and adds it to the RenderingCode AST.
Then, the system displays a menu to the user, allowing them to replace the Hole element with a RenderingCode based on the JSON value.
After the user chooses to add the new RenderingCode element, the system uses the previously defined \emph{recognizeJson} function to create the new element based on the corresponding JSON value.
Following that, the system uses the \emph{replace} function to replace the existing Hole element at the specified path with the newly created RenderingCode.
Lastly, the system traverses the modified RenderingCode AST and JSON AST as described in Section~\ref{sec:traversal} and performs operations such as displaying modification menus and element preview.

\begin{figure}[htbp]
	\centering
	\begin{tikzpicture}[node distance=1.5cm and 1.5cm, auto,
			user/.style={rectangle, draw=black, thick, fill=yellow!20,
					text width=7em, text centered, rounded corners, minimum height=3em},
			system/.style={rectangle, draw=black, thick, fill=blue!20,
					text width=7em, text centered, rounded corners, minimum height=3em},
			legend/.style={rectangle, draw=black, thin, fill=white,
					text width=4em, text centered,rounded corners, minimum height=2em, font=\footnotesize},
			line/.style={draw, thick, -stealth', shorten >=2pt, shorten <=2pt}]

		% Define nodes
		\node [system] (concrete) {New JSON value visited};
		\node [system, right=of concrete] (hole) {Create a Hole};
		\node [system, right=of hole] (holemenu) {Display Hole Menu};
		\node [user, below=of holemenu] (replace) {Replace Hole};
		\node [system, left=of replace] (recognize) {Create new RenderingCode based on JSON value};
		\node [system, left=of recognize] (update) {Update AST with new RenderingCode};
		\node [system,  below=of update] (final) {Traverse the Newly modified RenderingCode AST};

		% Draw connections
		\path [line] (concrete) -- (hole);
		\path [line] (hole) -- (holemenu);
		\path [line] (holemenu) -- (replace);
		\path [line] (replace) -- (recognize);
		\path [line] (recognize) -- (update);
		\path [line] (update) -- (final);
		% Add step numbers and labels
		\foreach [count=\i] \n/\l in {concrete/System, hole/System, holemenu/System, replace/User,recognize/System, update/System,  final/System}
			{
				\node[font=\small, above left, text=black] at (\n.north west) {\i};
				\node[font=\tiny, below right, text=black] at (\n.south east) {\l};
			}

		% Compact Legend
		\node[legend, fill=yellow!20,  right=1cm of final] (user_legend) {User Operation};
		\node[legend, fill=blue!20, right=0.5cm of user_legend] (system_legend) {System Operation};

	\end{tikzpicture}\caption{Single RenderingCode Creation Process}
	\label{fig:element-creation}
\end{figure}

\subsection{Combining RenderingCode and JSON data}
Once the RenderingCode AST is created, the system needs to combine the RenderingCode elements with the data from the JSON AST.
We use \emph{Structural referencing} to combine a RenderingCode element with the corresponding JSON value.
By Structural referencing we describe a process where a RenderingCode with a specific assigned Path is combined with a JSON value with the same Path.
We can do this easily, as the structure of the RenderingCode AST mirrors that of the JSON AST, with the exception of CustomWrappers and CustomElements,
which are ignored during the process and we preserve the paths of wrapped RenderingCode elements.
Thanks to the approach of Structural referencing, we can perform this process during our previously described traversal in Section~\ref{sec:traversal}.


\section{Summary}

In this Chapter, we described the the core logic and domain of our programming system.
This core logic serves as the foundation of the prototype programming system, which we will describe in the following Chapter.
\begin{itemize}
	\item In Section~\ref{sec:hole-based}, we defined the Hole-based approach to creating UI elements based on concrete data.

	\item In Section~\ref{sec:types}, we defined the Model of the application consisting of types representing the provided data and UI elements.
	      We also described the mapping between the input data and our internal representation of created UI elements.

	\item In Section~\ref{sec:creation}, we described the \emph{Incremental Creation Process} including the User and System operation.
	      Then, we described the simultanious traversal of the JSON AST and RenderingCode AST.
	      After that, we described the process of replacing and modifying the RenderingCode AST using a Path-based approach.
	      Lastly, we described the process of combining the RenderingCode elements with correspond JSON values.
\end{itemize}
