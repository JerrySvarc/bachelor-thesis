\chapter{Design principles}
\label{chap:design}

This chapter describes the main design principles behind the data-driven low-code programming system for creating web applications shown in Chapter~\ref{chap:walktrough}.
The main design principles discussed include:
\begin{enumerate}
	\item Data-driven approach
	\item Low-code editing
	\item Code generation capability
\end{enumerate}

\section{Data-driven approach}

The main general design principle of the proposed programming system is the \emph{data-driven} approach in context of creating web application UI elements.
We define the general approach as follows:
\begin{defn}[Data-driven approach]
	A software development approach where the creation of a particular software artifact is dependant on \textbf{concrete data} provided to the system.
\end{defn}

Meaning that the development process is tied to the specific data provided.
In the context of creating web application UI elements, some data can be used only for certain UI elements as to maintain the correct semantic meaning and functionality of the elements.
Another definition which could be used to describe our approach is the following:

\begin{defn}[Data to UI]
	Defined by \citet{Sahay_Indamutsa_Di} as "A data-driven approach that starts from data modeling and then builds the user interface of the application followed by the specification of business logic rules and workflows."
\end{defn}

This definition also includes the concepts of data modeling and specification of business logic rules and workflows but we only focus on the aspect of user interface creation in this thesis.
The main difference between our programming system and other systems described by \citet{Sahay_Indamutsa_Di} as \emph{Data to UI} such as \citet{mendix} is that creating the user interface is directly tied to the data provided to the application.
Usually, the \emph{Data to UI} systems provide options to create pre-made UI elements, which may then reference and use the provided data.
Whereas our proposed system creates the UI elements based on the data itself which then reference the provided data.

The system analyzes the provided data and provides development options based on its structure and properties.
The analysis can be tailored to the specific aims of the programming system.
For example, the proposed application for creating web applications can analyze the data and map it to specific HTML elements based on its type and structure.
Usually, the developer performs this data analysis and creates the GUI elements by hand.
Using this approach, the developer can offload some of this work onto the system and focus on other aspects of the development process.

The data analysis can employ various techniques, such as data structure and data type recognition.
One option is that the system performs the analysis once after the developer provides data to the system, and the system provides options based on this single analysis.
Alternatively, the system can perform it continuously throughout the development process and provide development options based on the already created application elements, similar to Output-directed programming described in Section~\ref{sec:odp}.
The system could also use various machine-learning techniques to improve the data analysis.

Another valuable property of systems employing this approach is their ability to be used in conjunction with other existing technologies, mainly those focused on data modeling and retrieval.
For example, the developer can use the proposed system employing this approach together with existing technologies such as \citet{graphql}.
Using GraphQL, the developer can specify the data they wish to use and its structure.
This data can then be provided to the system and used to create the desired web application.




\section{Low-code editor}

The second design principle is the concept of a \emph{low-code editor}.
Low-code editors usually provide a graphical user interface through which the developer interacts with the system and develops the desired application.
The system provides contextual menus to aid in the creation of the application.
Various existing low-code programming editors were presented previously in section \ref{sec:low-code}.

The visual design and functionality of the system is inspired by the "what you see is what you get" (abbreviated as \emph{WYSIWYG}) editors presented by \citet{Yang_Gupta_Botev_Churchill_Levchenko_Shanmugasundaram_2008}.
This sub-category of low-code editors displays the application's output as if running inside its native environment.
This design allows the system to update the output every time a change is made to the program and display its result, allowing for a more dynamic development loop \cite{output-directed-programming}.

Our prototype system interweaves the contextual menus with the application's output, as seen in chapter \ref{chap:walktrough}.
Each menu is placed next to the corresponding element, which it can modify.
The developer can also hide these menus and show only the application's output.
The design of the contextual menus was influenced by menus in currently existing programming systems, such as property menus from the \emph{Visual studio} editor.

\section{Code generation}

The last design principle is the ability to generate a plain−text representation of the created application.
This ability addresses some of the concerns and main pitfalls of many low-code programming systems, as stated by \citet{Pinho_Aguiar_Amaral_2023}.
These concerns include \emph{Interoperability issues} and \emph{Vendor lock-in}.
Both are mitigated as the user can download the already created part of the application in its textual representation and continue its modification in another programming system.

Our prototype implementation also provides a code generation capability.
The type system of the application represents the created parts of the application in a form that allows translation of the created application to various programming languages and other technologies.
The prototype system can target the \emph{PHP} programming language but could be extended to provide a wider selection of target technologies.













