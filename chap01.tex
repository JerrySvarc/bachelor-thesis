\chapter{Background}
\label{chap:background}

\section{Low-code development systems}


\xxx{Low-code development is defined by \citet{Pinho_Aguiar_Amaral_2023} as \emph{"an approach for software development that uses tools that minimize (or eliminate) the number of lines of code written."}
	This definition accommodates a broad spectrum of tools and development practices, such as visual programming tools,
	integrated development environments with code generation capabilities, development tools powered by artificial intelligence, and generators that generate source code based on template files.
	In this thesis, we will mainly focus on visual programming tools. }

\xxx{Visual programming tools present a graphical user interface to the user. This graphical user interface serves as an abstraction over
	the development domain. It provides tools to create and modify software elements. }

\subsection{Hypercard}
\xxx{
	\emph{Hypercard} is a low-code development system created by Bill Atkinson for the Macintosh operating system. Apple released the program in 1987 at the
	Macworld exposition in Boston~\cite{hyper_release}. Apple developed and maintained the program until 1998. The program became very popular, and similar programs and clones of Hypercard were created after the discontinuation.
}

\xxx{The following summary of functionality is based on a book written by~\citet{goodman_hypertext}.
	The fundamental elements that the user creates are called \emph{cards}. Cards can hold data as text, have custom graphics, contain buttons, and implement custom behavior.
	Users can implement the custom behavior using a scripting language called \emph{HyperTalk}. Then, users can group cards into \emph{stacks}. A stack is a collection of cards with the same type of information.
	The program saves a stack as a single file to the disk. Finally, users can distribute and modify these stacks.
}

\begin{figure}
	\centering
	\includegraphics[width=.7\linewidth]{img/hypercard−menu.pdf}
	\caption{Selection of a user level in the Hypercard preferences menu.}
	\label{fig:f}
\end{figure}

\xxx{
	The program offers users a choice of a user level, as seen in figure \ref{fig:f}. The program changes the user interface's capabilities based on the selected user level.
	The higher the selected user level, the more options the program enables and displays. A selected level enables all previous-level options alongside other options and functionality.
	Users can choose from five different user levels:
	\begin{enumerate}
		\item Browsing - enables no modification of cards and is mainly intended to be used to view the different cards inside a stack.
		\item Typing - enables inserting and modifying text data inside the cards.
		\item Painting - enables the creation of custom graphics inside the cards. The program provides different graphical options and tools.
		\item Authoring - adds the ability to add buttons and fields. This way, the users can add card functionality without writing code.
		\item Scripting - adds the ability to use the HyperTalk scripting language to modify the behavior of the different elements inside the cards.
	\end{enumerate}
}

\xxx{
	The program became popular with users at many different skill levels thanks to separating more advanced features from the basic ones.
	Less advanced users can learn to use the interface more easily and gradually increase their proficiency with the program without being overwhelmed by options.
	On the other hand, advanced users can set the highest user level and use the program fully from the start. }


\subsection{Darklang}
\xxx{\emph{Darklang}\footnote{\url{https://darklang.com/} [visited on 2024-04-22][online]} is a cloud-based low-code programming system for building web application backends created by Paul Biggar.
	The original system's development was discontinued in 2023 and renamed Darklang-classic.
}

\xxx{The following summary follows the official Darklang~documentation\footnote{\url{https://docs.darklang.com/} [visited on 2024-04-22][online]}.
	The Darklang project consists of a low-code editor and a programming language. It also provides tools for creating persistent storage and for deployment.
	The editor provides the main functionality of the service. It consists of a canvas that displays the created application elements as draggable boxes.
	Each box provides options to modify the element. These options include different menus and input fields.
	Some elements can be modified using the Darklang programming language to implement custom behavior.
	The user can create program elements of the following categories:
	\begin{itemize}
		\item HTTP handlers - definition of API endpoints
		\item Persistent storage - database creation and modification
		\item Workers - processing of background tasks
		\item Cron jobs - scheduled jobs with custom behavior
		\item REPL - a general-purpose programmable element
		\item Functions - a reusable element with custom behavior
  \end{itemize}
	Through the combination of these elements, the users can create application backends of varying complexity.
	After the user creates and modifies a particular element, the system automatically redeploys the entire application.
	This allows for an interactive style of development where changes made to the system are almost immediately visible.}

\xxx{Users can implement custom functionality by using the \emph{Darklang language}.
	It is a statically-typed \emph{ML-style} language similar to \emph{OCaml}\footnote{\url{https://ocaml.org/} [visited on 2024-04-23][online]}.
	Some of the features of the language include:
	\begin{itemize}
		\item The pipe operator.
		\item Functions for working with collections, such as the \emph{map} and \emph{fold} functions.
		\item Pattern matching.
		\item Implicit returns.
	\end{itemize}
}

\subsection{Output-directed programming}
\xxx{
	\emph{Output-directed programming} is described by \citet{sketch-and-sketch} as
	a programming paradigm where users construct plain text programs using mouse-based operations on the program's output.
	The text representation of the program serves as the main representation.
	This means that the user can also directly change the text of the program, which will result in a change in output.
	This allows the user to make changes that are difficult or impossible to make using the direct manipulation of the program's output.
	This bimodal approach can be seen as an extension of the standard programming paradigm where the manipulation of the text-based program
	representation is the only way to change the program's output.
}

\xxx{One example of a programming system that demonstrates this paradigm is called \emph{Sketch-and-Sketch} firstly presented by \citet{output-directed-programming}.
	The system allows the creation and modification of Scalable Vector Graphics using direct manipulation of the program's SVG output.
}

\section{Web development tools}
\subsection{Elm}
\subsection{Web development using F\# }
