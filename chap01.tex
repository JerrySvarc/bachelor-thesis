\chapter{Background}
\label{chap:background}

\section{Low-code development systems}


\xxx{Low-code development is defined by \citet{Pinho_Aguiar_Amaral_2023} as \emph{"an approach for software development that uses tools that minimize (or eliminate) the number of lines of code written."}
This definition accommodates a broad spectrum of tools and development practices, such as visual programming tools, 
integrated development environments with code generation capabilities, development tools powered by artificial intelligence, and generators that generate source code based on template files.
In this thesis, we will mainly focus on visual programming tools. }

\xxx{Visual programming tools present a graphical user interface to the user. This graphical user interface serves as an abstraction over 
the development domain. It provides tools to create and modify software elements. }

\subsection{Hypercard}
\xxx{
\emph{Hypercard} is a low-code development system created by Bill Atkinson for the Macintosh operating system. Apple released the program in 1987 at the 
Macworld exposition in Boston~\cite{hyper_release}. Apple developed and maintained the program until 1998. The program became very popular, and similar programs and clones of Hypercard were created after the discontinuation. 
 }

\xxx{The following summary of functionality is based on a book\footnote{The book serves as a comprehensive guide for the program.} written by~\citet{goodman_hypertext}.
The fundamental elements that the user creates are called \emph{cards}. Cards can hold data as text, have custom graphics, contain buttons, and implement custom behavior.
Users can implement the custom behavior using a scripting language called \emph{HyperTalk}. Then, users can group cards into \emph{stacks}. A stack is a collection of cards with the same type of information. 
The program saves a stack as a single file to the disk. Finally, users can distribute and modify these stacks. 
}

\xxx{
The program offers users a choice of a user level. The program changes the user interface's capabilities based on the selected user level. 
The higher the selected user level, the more options the program enables and displays. A selected level enables all previous-level options alongside other options and functionality.
Users can choose from five different user levels:
\begin{enumerate}
    \item Browsing - enables no modification of cards and is mainly intended to be used to view the different cards inside a stack.
    \item Typing - enables inserting and modifying text data inside the cards.
    \item Painting - enables the creation of custom graphics inside the cards. The program provides different graphical options and tools.  
    \item Authoring - adds the ability to add buttons and fields. This way, the users can add card functionality without writing code.
    \item Scripting - adds the ability to use the HyperTalk scripting language to modify the behavior of the different elements inside the cards. 
\end{enumerate}
}
\xxx{
The program became popular with users at many different skill levels thanks to separating more advanced features from the basic ones. Less advanced users can learn to 
use the interface more easily and gradually increase their proficiency with the program without being overwhelmed by options. On the other hand, advanced users can set the highest 
user level and use the program fully from the start. }


\subsection{Darklang}
\xxx{Darklang is a low-code programming cloud-based system for creating web application backends.   }

\subsection{}


\subsection{Output-directed programming}
\xxx{Output-directed programming}


\section{Web development tools}
\subsection{Elm}
\subsection{Web development using F\# }










