%%% Please fill in basic information on your thesis, which will be automatically
%%% inserted at the right places.

% Type of your thesis:
%	"bc" for Bachelor's
%	"mgr" for Master's
%	"phd" for PhD
%	"rig" for rigorosum
\def\ThesisType{bc}

% Language of your study programme:
%	"cs" for Czech
%	"en" for English
\def\StudyLanguage{cs}

% Thesis title in English (exactly as in the official assignment)
% (Note: \xxx is a "ToDo label" which makes the unfilled visible. Remove it.)
\def\ThesisTitle{Data-driven low-code programming system}

% Author of the thesis (you)
\def\ThesisAuthor{Jaroslav Švarc}

% Year when the thesis is submitted
\def\YearSubmitted{2025}

% Name of the department or institute, where the work was officially assigned
% (according to the Organizational Structure of MFF UK in English,
% see https://www.mff.cuni.cz/en/faculty/organizational-structure,
% or a full name of a department outside MFF)
\def\Department{Department of Distributed and Dependable Systems}

% Is it a department (katedra), or an institute (ústav)?
\def\DeptType{Department}

% Thesis supervisor: name, surname and titles
\def\Supervisor{Mgr. Tomáš Petříček, Ph.D.}

% Supervisor's department (again according to Organizational structure of MFF)
\def\SupervisorsDepartment{Department of Distributed and Dependable Systems}

% Study programme (does not apply to rigorosum theses)
\def\StudyProgramme{Computer science}

% An optional dedication: you can thank whomever you wish (your supervisor,
% consultant, who provided you with tea and pizza, etc.)
\def\Dedication{
	\xxx{Dedication.}
}

% Abstract (recommended length around 80-200 words; this is not a copy of your thesis assignment!)
\def\Abstract{%
	Creating client-side web applications using low-code programming systems is increasingly popular. Traditional low-code development systems follow a UI-to-data approach where developers create user interface elements and populate them with data. However, the data-to-UI approach, where the development process begins with concrete data that drives the creation of corresponding UI elements, remains largely unexplored. Using our InterfaceSmith programming system, we demonstrate how this data-driven approach enables incremental UI creation through a hole-based approach where UI elements are created based on the provided data structure. The system aids developers in modifying the interface through context menus and behavior modification options inspired by the Elm architecture. Our evaluation through benchmarks, including a TODO list application and tasks from the 7GUIs benchmark suite, demonstrates the system's effectiveness in reducing the amount of code developers need to write while maintaining the ability to implement complex web application functionality.
}

% 3 to 5 keywords (recommended) separated by \sep
% Keywords are useful for indexing and searching for the theses by topic.
\def\ThesisKeywords{
	low-code programming\sep programming systems \sep data-driven programming}

% If any of your metadata strings contains TeX macros, you need to provide
% a plain-text version for use in XMP metadata embedded in the output PDF file.
% If you are not sure, check the generated thesis.xmpdata file.
\def\ThesisAuthorXMP{\ThesisAuthor}
\def\ThesisTitleXMP{\ThesisTitle}
\def\ThesisKeywordsXMP{\ThesisKeywords}
\def\AbstractXMP{\Abstract}

% If your abstracts are long and do not fit in the infopage, you can make the
% fonts a bit smaller by this setting. (Also, you should try to compress your abstract more.)
\def\InfoPageFont{}
%\def\InfoPageFont{\small}  % uncomment to decrease font size

% If you are studing in a Czech programme, you also need to provide metadata in Czech:
% (in English programmes, this is not used anywhere)

\def\ThesisTitleCS{Data-driven low-code programming system}
\def\DepartmentCS{Katedra distribuovaných a spolehlivých systémů}
\def\DeptTypeCS{Katedra}
\def\SupervisorsDepartmentCS{Katedra distribuovaných a spolehlivých systémů}
\def\StudyProgrammeCS{Informatika}

\def\ThesisKeywordsCS{%
	low-code programování\sep vývojové nástroje\sep datově řízené programování}

\def\AbstractCS{%
	\xxx{cesky abstrac}
}
