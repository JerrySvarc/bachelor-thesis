%%% Please fill in basic information on your thesis, which will be automatically
%%% inserted at the right places.

% Type of your thesis:
%	"bc" for Bachelor's
%	"mgr" for Master's
%	"phd" for PhD
%	"rig" for rigorosum
\def\ThesisType{bc}

% Language of your study programme:
%	"cs" for Czech
%	"en" for English
\def\StudyLanguage{cs}

% Thesis title in English (exactly as in the official assignment)
% (Note: \xxx is a "ToDo label" which makes the unfilled visible. Remove it.)
\def\ThesisTitle{Data-driven low-code programming system}

% Author of the thesis (you)
\def\ThesisAuthor{Jaroslav Švarc}

% Year when the thesis is submitted
\def\YearSubmitted{2024}

% Name of the department or institute, where the work was officially assigned
% (according to the Organizational Structure of MFF UK in English,
% see https://www.mff.cuni.cz/en/faculty/organizational-structure,
% or a full name of a department outside MFF)
\def\Department{Department of Distributed and Dependable Systems}

% Is it a department (katedra), or an institute (ústav)?
\def\DeptType{Department}

% Thesis supervisor: name, surname and titles
\def\Supervisor{Mgr. Tomáš Petříček, Ph.D.}

% Supervisor's department (again according to Organizational structure of MFF)
\def\SupervisorsDepartment{Department of Distributed and Dependable Systems}

% Study programme (does not apply to rigorosum theses)
\def\StudyProgramme{Computer science}

% An optional dedication: you can thank whomever you wish (your supervisor,
% consultant, who provided you with tea and pizza, etc.)
\def\Dedication{%
	\xxx{Dedication.}
}

% Abstract (recommended length around 80-200 words; this is not a copy of your thesis assignment!)
\def\Abstract{%
	This thesis combines the Low-code and Data-driven approaches to software development within a single programming system for creating client-side web applications. We present Data-driven UI, a prototype system where UI elements are created based on concrete data.
	The system implements core design principles, including data-driven approach, low-code editing, and code generation capabilities.
	The system employs a hole-based approach for incremental UI creation and structural referencing to combine UI elements with input data. We evaluate the system through benchmarks, including a To-Do application and several tasks from the 7GUIs benchmark. The approach demonstrates strengths in guided UI creation, mitigation of vendor lock-in, and improved interoperability with existing systems. However, limitations include the need for upfront data preparation and potential performance concerns for large-scale applications.
}

% 3 to 5 keywords (recommended) separated by \sep
% Keywords are useful for indexing and searching for the theses by topic.
\def\ThesisKeywords{
	low-code programming\sep programming systems \sep data-driven programming}

% If any of your metadata strings contains TeX macros, you need to provide
% a plain-text version for use in XMP metadata embedded in the output PDF file.
% If you are not sure, check the generated thesis.xmpdata file.
\def\ThesisAuthorXMP{\ThesisAuthor}
\def\ThesisTitleXMP{\ThesisTitle}
\def\ThesisKeywordsXMP{\ThesisKeywords}
\def\AbstractXMP{\Abstract}

% If your abstracts are long and do not fit in the infopage, you can make the
% fonts a bit smaller by this setting. (Also, you should try to compress your abstract more.)
\def\InfoPageFont{}
%\def\InfoPageFont{\small}  % uncomment to decrease font size

% If you are studing in a Czech programme, you also need to provide metadata in Czech:
% (in English programmes, this is not used anywhere)

\def\ThesisTitleCS{Data-driven low-code programming system}
\def\DepartmentCS{Katedra distribuovaných a spolehlivých systémů}
\def\DeptTypeCS{Katedra}
\def\SupervisorsDepartmentCS{Katedra distribuovaných a spolehlivých systémů}
\def\StudyProgrammeCS{Informatika}

\def\ThesisKeywordsCS{%
	\xxx{klíčová slova\sep klíčové fráze}
}

\def\AbstractCS{%
	\xxx{Abstrakt práce přeložte také do češtiny.}
}
